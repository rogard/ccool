\documentclass{article}
\usepackage{amsmath,amssymb}
\usepackage[english]{babel}
\usepackage[style=numeric, maxnames=999, backend=biber]{biblatex}
\begin{filecontents}{\jobname.bib}
@misc{ccool,
        author = {Erwann Rogard},
        title = {The \texttt{ccool} package},
        year = {2020},
        url = {https://github.com/rogard/ccool}
}
\end{filecontents}
\addbibresource{\jobname.bib}
\usepackage{datetime2}
\DTMsetup{datesep=/,style=mmddyyyy}
\usepackage{enumitem}
\usepackage{fancyvrb}
\usepackage{fvextra}
\usepackage[T1]{fontenc}
\usepackage{ccool}
\usepackage{hyperref}

\providecommand\docfillblank{\begin{minipage}[t]{\linewidth}\end{minipage}}
\providecommand\docways[1]{way~\ref{ways:#1}}
\providecommand\docWays[1]{Way~\ref{ways:#1}}

\title{A gentle intro to \texttt{ccool} for \LaTeX}

\author{Erwann Rogard\thanks{firstname dot lastname AusTria gmail dot com}}

\date{\today}

\begin{document}
\maketitle

\begin{abstract}
  Breaking down the example in the abstract of the package's doc\cite{ccool}
\end{abstract}


Let's say we want to typeset:
\begin{center}
  ``\Ccool<Math>[Let~]i{\mathbb{#1}}{Nat = N, Real = R}*s{{~\rm{and}~}}[~denote the natural and real numbers.]{}''
\end{center}

There are at least four ways to do it:
\begin{enumerate}[label=\emph{\roman*)}]
\item \label{ways:i}
  \begin{Verbatim}[breaklines=true]
    Let~$\mathbb{N}$ and $\mathbb{R}$ denote the natural and real numbers.
  \end{Verbatim}
\item   \label{ways:ii}
  \begin{Verbatim}[breaklines=true]
    \NewDocumentCommand\Nat{}{\mathbb{N}}
    \NewDocumentCommand\Real{}{\mathbb{R}}
    Let~$\Nat$ and $\Real$ denote the natural and real numbers.
  \end{Verbatim}
%    \NewDocumentCommand\Nat{}{\mathbb{N}}
%    \NewDocumentCommand\Real{}{\mathbb{R}}
%    Let~$\Nat$ and $\Real$ denote the natural and real numbers.

\item  \label{ways:iii}
  \begin{Verbatim}[breaklines=true]
    \Ccool i{\mathbb{#1}}{ Nat = N, Real = R }
    Let~$\Nat$ and $\Real$~denote the natural and real numbers.
  \end{Verbatim}
%    \Ccool i{\mathbb{#1}}{ Nat = N, Real = R }
%    Let~$\Nat$ and $\Real$~denote the natural and real numbers.

\item  \label{ways:iv}
  \begin{Verbatim}[breaklines=true]
    \Ccool[Let~]
    i{\mathbb{#1}}{ Nat = N, Real = R }*s{{~\rm{and}~}}
    [~denote the natural and real numbers.]{}
  \end{Verbatim}
\end{enumerate}

\docWays{i} is prone to errors, should the author change \verb+\mathbb{R}+ to \verb+\mathcal{R}+ throughout the document.
\docWays{ii} corrects that, as the change need only be made in place.
Also, it has the advantage that it makes the meaning of \verb+\mathbb{R}+ explicit (real numbers), in the source file.
The advantage of \docways{iii} over  \docways{iv}, is that it is less verbose,
and, in this case, that it eliminates the redundancy of \verb|\mathbb|.
By expanding the macro definitions (\verb+\Real+) as they are made, \docways{iv} allows to
make them blend with the text, which some authors may find desirable.

NB: The features covered are not exhaustive.

This document was generated using~\CcoolVers.

\phantomsection
\addcontentsline{toc}{section}{References}
\printbibliography[heading=subbibliography]

\end{document}
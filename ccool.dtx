% \iffalse meta-comment
% !TEX program  = pdfLaTeX
%<*internal> 
\iffalse
%</internal> 
%<*readme> 
----------------------------------------------------------------
##### ccool --- A key-value interface for generating commands
- Source repository: https://github.com/rogard/ccool
- Released under the LaTeX Project Public License v1.3c or later
- See http://www.latex-project.org/lppl.txt
----------------------------------------------------------------

%</readme> 
%<*internal> 
\fi
\def\nameofplainTeX{plain}
\ifx\fmtname\nameofplainTeX\else
\expandafter\begingroup
\fi
%</internal> 
%<*install> 
\input docstrip.tex
\keepsilent
\askforoverwritefalse
\preamble
----------------------------------------------------------------------------
ccool --- A key-value interface for generating commands
Released under the LaTeX Project Public License v1.3c or later
See http://www.latex-project.org/lppl.txt
----------------------------------------------------------------------------

\endpreamble
\postamble

Copyright (C) 2020 by Erwann Rogard

This work may be distributed and/or modified under the
conditions of the LaTeX Project Public License (LPPL), either
version 1.3c of this license or (at your option) any later
version.  The latest version of this license is in the file:

http://www.latex-project.org/lppl.txt

This work is "maintained" (as per LPPL maintenance status) by
Erwann Rogard.

This work consists of the file ccool.dtx and the derived files:
ccool.sty, and ccool.pdf.

\endpostamble
\generate{
  \file{\jobname.sty}{\from{\jobname.dtx}{package}}
}
%</install> 
%<install> \endbatchfile
%<*internal> 
\generate{
  \file{\jobname.ins}{\from{\jobname.dtx}{install}}
}
\nopreamble\nopostamble
\generate{
  \file{README.md}{\from{\jobname.dtx}{readme}}
}
\ifx\fmtname\nameofplainTeX
\expandafter\endbatchfile
\else
\expandafter\endgroup
\fi
%</internal> 
%<package> \NeedsTeXFormat{LaTeX2e}[2020/02/02]
%<package> \RequirePackage{etoolbox}[2019/09/21]
%<package> \RequirePackage{l3keys2e}[2020/03/06]
%<package> \RequirePackage{xparse}[2020/03/06]
%<package> \ProvidesExplPackage
%<package> {ccool}                                                                                      % Package name
%<package> {2020/04/17}                                                                                 % Release date
%<package> {2.2}                                                                                        % Release version
%<package> {cool --- A key-value interface for generating commands}                                     % Description
% 
%<*driver> 
\documentclass[full]{l3doc}
\listfiles
\usepackage{amsmath, amssymb}
\usepackage[english]{babel}
\usepackage{bookmark}
\usepackage{ccool}
\usepackage{fvextra}% csquotes should be loaded after fvextra
\usepackage[T1]{fontenc}% \char`[
\usepackage{pdfpages}
\usepackage{tabto}
\usepackage{tcolorbox}
\tcbuselibrary{listings, breakable}
\makeatletter
\newcommand*{\docsetnameref}{\def\@currentlabelname}%https://tex.stackexchange.com/questions/537751
\makeatother
\ExplSyntaxOn
\tl_gset:Nn \partname {Part}
\ExplSyntaxOff
\EnableCrossrefs
\CodelineIndex
\RecordChanges
% ^^A\AtEndDocument { \PrintChanges \PrintIndex }
\ExplSyntaxOn
\providecommand\docarg[1]{\texttt{#1}} % fun[param] (macro) vs fun[arg] (eval)
\providecommand\docargnoval{\c_novalue_tl}
\providecommand\docassign[2]{#1~$\leftarrow$~#2}
\providecommand\docccept[1]{\textit{#1}}
\providecommand\doccceptbool{boolean}
\providecommand\doccceptcode{code}
\providecommand\doccceptint{integer}
\providecommand\doccceptgroup{local~group}
\providecommand\doccceptkvl{keyval~list}
\providecommand\doccceptpath{path}
\providecommand\doccceptpre{preamble}
\providecommand\docccepttok{token}
\providecommand\docenvdoc{\env{document}}
\providecommand\docdefaultfor{default~for~}
\providecommand\doceval[1]{\texttt{\char`\{}#1\texttt{\char`\}}}
\providecommand\docfillblank{\begin{minipage}[t]{\linewidth}\end{minipage}}
\providecommand\docissuedo{Do: }
\providecommand\docissuedont{Don't: }
\providecommand\docissuesymp{Symptom: }
\providecommand\doclist[1]{Listing~\ref{listing:#1}}
\providecommand\docopto[1]{\texttt{[}#1\texttt{]}}
\providecommand\docopte[2]{\texttt{#1}\doceval{#2}}
\providecommand\docoptd[1]{\texttt{\textless}#1\texttt{\textgreater}}
\providecommand\docpipe{\textbar}
\cs_new:Nn \__ccool_docu:n{\MakeUppercase #1}
\providecommand\docstep[1]{step~\ref{step:#1}}
\providecommand\docsee{See:~}
\providecommand\docccepttl{token~list}
\providecommand\doctip{\noindent\textbf{Tip}:~}
\providecommand\docU[1]{\exp_args:Nx \__ccool_docu:n{#1}}
\providecommand\docvers[2]{v#1.#2}
\providecommand\docwarn{\noindent\textbf{Warning}:~}
\providecommand\pkgparap{\texttt{+}}%append
\providecommand\pkgparde{kvl$_{1}$}
\providecommand\pkgpardenxt{kvl$_{2}$}
\providecommand\pkgparex{\texttt{*}}%expand
\providecommand\pkgparhe{tl$_{1}$}%head
\providecommand\pkgparin{code$_{1}$}%inner
\providecommand\pkgparpa{tl$_{2}$}%param
\providecommand\pkgparou{code$_{2}$}%outer
\providecommand\pkgpars{\Arg{\pkgparsi}\docpipe\Arg{\pkgparsi}\Arg{\pkgparsii}\docpipe\Arg{\pkgparsi}\Arg{\pkgparsii}\Arg{\pkgparsiii}}%separ
\providecommand\pkgparsi{tl$_{3}$}
\providecommand\pkgparsii{tl$_{4}$}
\providecommand\pkgparsiii{tl$_{5}$}
\providecommand\pkgparta{tl$_{6}$}%tail
\providecommand\pkgkey{key$_{i}$}
\providecommand\pkgval{val$_{i}$}
\providecommand\pkgoptex{\docarg{Expans}}
\providecommand\pkgoptfi{\docarg{File}}
\providecommand\pkgoptin{\docarg{Inner}}
\providecommand\pkgoptpa{\docarg{Param}}
\providecommand\pkgoptpad{\docarg{Default}}%default
\providecommand\pkgoptou{\docarg{Outer}}
\providecommand\pkgoptwr{\docarg{Write}}
\providecommand\pkgoptse{\docarg{Separ}}
\providecommand\pkgobj[1]{object identified by #1}
\providecommand\pkgoptions{kvl0}
\providecommand\pkgsep[1]{\Arg{#1}}
\providecommand\pkgwrite{\cs{Ccool}\docarg{\meta{\pkgparpa}}\docopte{i}{\meta{\pkgparin}}\Arg{\pkgparde}}
\ExplSyntaxOff
\begin{document}
\DocInput{\jobname.dtx}
\end{document}
%</driver> 
% \fi
% 
% \GetFileInfo{\jobname.sty}
% \begin{documentation}
%   \title{The \pkg{ccool} package\thanks{^^A
%   This file describes version \fileversion, last revised \filedate.^^A
% }^^A
% }
%   \author{Erwann Rogard\thanks{firstname dot lastname AusTria gmail dot com}}
%   
%   \date{Released \filedate}
%   
%   \maketitle
%
%   \begin{abstract} \pkg{ccool} is, at its core, a \textit{key-value} interface built upon \pkg{xparse}\cite{xparse},
%     for generating commands, with a view to reduce the overhead\footnote{See \autoref{other:geneal} for the initial motivation}.
%     Specifically, \nameref{usage:cs:ccool}\Arg{\doccceptkvl} creates for each \textit{i}
%     the command  \cs{\meta{\pkgkey}}, with \meta{\pkgval} as its implementation.
%     For instance, the side effect of |{ Real = \mathbb{R} }|
%     is that |\Real| expands to $\mathbb{R}$.
%     A \docccept{\doccceptkvl} can optionally be expanded in place (append \pkgparex), according to default or inline rules,
%     and interspersed with optional arguments that are destined exclusively for expansion
%     (|[Let~]| and |[~denote real numbers.]|). Thus, in theory,
%     a document could be typeset, starting with a single \nameref{usage:cs:ccool} declaration.
%     An optional parameter prepended to the \docccept{\doccceptkvl}, \docoptd{\meta{param}},
%     allows to parameterize the keys (for instance, one parameter value for the style, another for a property).
%     In conjunction with lamba expressions,
%     the commands created through this interface can accept arguments the same way as \cs{NewDocumentCommand}.
%     A possible use of this, in Math,
%     could be to encode the way \hyperref[listing:lambda:i]{functions and operators} are typeset,
%     (say $(.)$ and $[.]$ around their arguments).
%     Optionally, the macros can be written to a file, and read, which
%     can be useful for typesetting documents sharing the same notation. 
%   \end{abstract}
%   
%   \tableofcontents 
%   
%   \part{Usage}\label{part:usage}
%   ^^A   \VerbatimFootnotes
%   
%   \setcounter{section}{0}
%   \label{usage:conv}
%   \addcontentsline{toc}{section}{\protect\numberline{\thesection}Convention}
%   \section*{Convention}
%   \begin{enumerate}
%   \item Loosely, those of \cite{interface3} and \cite{xparse}, for example as to the meaning of \meta{\docccepttl}. 
%   \item If unspecified, the environment in which a macro must be declared is \docenvdoc.
%   \end{enumerate}
%   
%   \refstepcounter{section}
%   \label{usage:load} 
%   \addcontentsline{toc}{section}{\protect\numberline{\thesection}Loading the package}
%   
%   \begin{function}{\usepackage}
%     \begin{syntax}
%       \cs{usepackage}\doceval{\pkg{ccool}}
%     \end{syntax}
%     \begin{description}
%     \item[Requirement]\docfillblank
%       \begin{enumerate}
%       \item \file{ccool.sty} is in the path of the \LaTeX~engine. See \autoref{part:other}, \autoref{other:support}.
%       \item Declare it in the~\docccept{\doccceptpre}
%       \end{enumerate}
%     \end{description}  
%   \end{function}
%   
%   \leavevmode
%   \refstepcounter{section}
%   \docsetnameref{\cs{Ccool}}
%   \label{usage:cs:ccool}
%   \addcontentsline{toc}{section}{\protect\numberline{\thesection}\cs{Ccool}}
%   \setcounter{subsection}{0}
%   \begin{function}{\Ccool}
%     \begin{syntax}
%       \cs{Ccool}
%       \docopto{\meta{\pkgparhe}}
%       \docoptd{\meta{\pkgparpa}}
%       \docopte{i}{\meta{\pkgparin}}
%       \Arg{\pkgparde}
%       \pkgparap
%       \pkgparex
%       \docopte{s}{\pkgpars}
%       \docopte{o}{\meta{\pkgparou}}
%       \docopto{\meta{\pkgparta}}
%     \end{syntax}
%     \begin{description}
%     \item[Requirement] \meta{\pkgparde} is specified (all others optional).
%     \end{description}
%   \end{function}
%   
%   \leavevmode
%   \refstepcounter{subsection}
%   \docsetnameref{\meta{\pkgparhe}}
%   \label{usage:par:he}
%   \addcontentsline{toc}{subsection}{\protect\numberline{\thesubsection}\docopto{\meta{\pkgparhe}}}
%   \DescribeOption{\meta{\pkgparhe}}
%   \begin{description}
%   \item[Example]|Let~|
%   \item[Semantics] Expands~\meta{\pkgparhe}
%   \end{description}
%   
%   \leavevmode
%   \refstepcounter{subsection}
%   \docsetnameref{\meta{\pkgparpa}}
%   \label{usage:par:pa}
%   \addcontentsline{toc}{subsection}{\protect\numberline{\thesubsection}\docoptd{\meta{\pkgparpa}}}
%   \DescribeOption{\meta{\pkgparpa}}
%   \begin{description}
%   \item[Default] \nameref{usage:opt:pa}'s
%   \item[Example] \pkgoptpad, |Style| and |Describe|, |ModelA| and |ModelB|
%   \end{description}
%   
%   \leavevmode
%   \refstepcounter{subsection}
%   \docsetnameref{\meta{\pkgparin}}
%   \label{usage:par:in}
%   \phantomsection\addcontentsline{toc}{subsection}{\protect\numberline{\thesubsection}\docopte{i}{\meta{\pkgparin}}}
%   \DescribeOption{\meta{\pkgparin}}
%   
%   \begin{description}
%   \item[Default] \nameref{usage:opt:in}'s
%   \item[Example] |\mathbb{#1}|
%   \end{description}
%   
%   \leavevmode
%   \refstepcounter{subsection}
%   \docsetnameref{\meta{\pkgparde}}
%   \label{usage:par:de}
%   \phantomsection\addcontentsline{toc}{subsection}{\protect\numberline{\thesubsection}\Arg{\pkgparde}}
%   \DescribeOption{\meta{\pkgparde}}
%   \begin{description}
%   \item[Example] |Real={\mathbb{R}}|
%   \item[Semantics]\docfillblank
%     \begin{enumerate}[label=\emph{\arabic*)}]
%       \setcounter{enumi}{0}
%     \item \docassign{\meta{\pkgval} }
%       { \meta{\pkgparin} applied to \meta{\pkgval} }\label{step:val}
%     \item \docassign{ \cs{\meta{\pkgkey}}\docoptd{\meta{\pkgparpa}}}{\meta{val_i}} defined in \docstep{val},
%       using \nameref{usage:opt:ex} for expansion. \label{step:key}
%     \item If \nameref{usage:opt:wr}, writes the input used by \docstep{key} to \nameref{usage:opt:fi}\label{step:write}
%     \end{enumerate}
%   \end{description}
%   
%   \leavevmode
%   \refstepcounter{subsection}
%   \docsetnameref{parameter \docarg{+}}
%   \label{usage:par:appto}
%   \phantomsection\addcontentsline{toc}{subsection}{\protect\numberline{\thesubsection}\docarg{+}}
%   \DescribeOption{+}
%   \begin{description}
%   \item[Semantics] Repeats \docstep{val}, \docstep{key}, and \docstep{write}, at \nameref{usage:cs:hook} (useful inside a \docccept{\doccceptgroup})
%   \end{description}
%   
%   \leavevmode
%   \refstepcounter{subsection}
%   \docsetnameref{parameter \docarg{*}}
%   \label{usage:par:ex}
%   \phantomsection\addcontentsline{toc}{subsection}{\protect\numberline{\thesubsection}\docarg{*}}
%   \DescribeOption{*}
%   
%   \begin{description}
%   \item[Semantics] Expands~\nameref{usage:par:ou} applied to the list created in \docstep{val},
%     using the separator specified by \nameref{usage:par:si}, \nameref{usage:par:sii}, and \nameref{usage:par:siii}.
%   \end{description}
%   
%   \leavevmode
%   \refstepcounter{subsection}
%   \docsetnameref{separators' parameter}
%   \label{usage:par:s}
%   \phantomsection\addcontentsline{toc}{subsection}{\protect\numberline{\thesubsection}\docopte{s}{\pkgpars}}
%   
%   \leavevmode
%   \refstepcounter{subsubsection}
%   \docsetnameref{\meta{\pkgparsi}}
%   \label{usage:par:si}
%   \DescribeOption{\meta{\pkgparsi}}
%   \begin{description}
%   \item[Default] \nameref{usage:opt:se}'s
%   \item[Example] |{~\in~}|
%   \end{description}
%   
%   \leavevmode
%   \refstepcounter{subsubsection}
%   \docsetnameref{\meta{\pkgparsii}}
%   \label{usage:par:sii}
%   \DescribeOption{\meta{\pkgparsii}}
%   \begin{description}
%   \item[Default] \nameref{usage:opt:se}'s 
%   \item[Example] |{,~}|
%   \end{description}
%   
%   \leavevmode
%   \refstepcounter{subsubsection}
%   \docsetnameref{\meta{\pkgparsiii}}
%   \label{usage:par:siii}
%   \DescribeOption{\meta{\pkgparsiii}}
%   \begin{description}
%   \item[Default] \nameref{usage:opt:se}'s
%   \item[Example] |{~\&~}|
%   \end{description}
%   
%   \leavevmode
%   \refstepcounter{subsection}
%   \docsetnameref{\meta{\pkgparou}}
%   \label{usage:par:ou}
%   \phantomsection\addcontentsline{toc}{subsection}{\protect\numberline{\thesubsection}\docopte{o}{\meta{\pkgparou}}}
%   \DescribeOption{\meta{\pkgparou}}
%   \begin{description}
%   \item[Default] \nameref{usage:opt:ou}'s
%   \item[Example] |$\left\{#1\right\}$| 
%   \end{description}
%   
%   \leavevmode
%   \refstepcounter{subsection}
%   \docsetnameref{\meta{\pkgparta}}
%   \label{usage:par:ta}
%   \phantomsection\addcontentsline{toc}{subsection}{\protect\numberline{\thesubsection}\docopto{\meta{\pkgparta}}}
%   \DescribeOption{\meta{\pkgparta}}
%   \begin{description}
%   \item[Semantics]
%     \nameref{usage:cs:ccool}\docopto{\meta{\pkgparta}}
%   \end{description}
%   
%   \leavevmode
%   \refstepcounter{section}
%   \docsetnameref{Other}
%   \label{usage:other}
%   \addcontentsline{toc}{section}{\protect\numberline{\thesection}Other}
%   \section*{Other}
%   Continued in \autoref{part:impl}, \autoref{impl:frontend}.
%   
%   \leavevmode
%   \refstepcounter{section}
%   \docsetnameref{Do's and dont's}
%   \label{usage:dosdont}
%   \addcontentsline{toc}{section}{\protect\numberline{\thesection}Do's and dont's}
%   \section*{Do's and dont's}
%   
%   \begin{enumerate}[label=\emph{\arabic*)}]
%   \item \docfillblank 
%     \begin{itemize}
%     \item[\docissuedont]
%       |\Ccool{ A = a, B = b }[Hello, world!]|
%     \item[\docissuedo]
%       |\Ccool{ A = a, B = b }[Hello, world!]{}| \newline or |\Ccool{ A = a, B = b } Hello, world!|
%     \end{itemize}
%   \item \docfillblank
%     \begin{itemize}
%     \item[\docissuedont] |$|\meta{\pkgkey}|<x$|.
%     \item[\docissuedo] |$|\cs{\meta{\pkgkey}}|{<}x$|
%     \end{itemize}
%   \item\docfillblank
%     \begin{itemize}
%     \item[\docissuedont] |[a, b)|
%     \item[\docissuedo] |{[}a, b{)}|
%     \end{itemize}
%   \item \docfillblank
%     \begin{itemize}
%     \item[\docissuedont] |\cal F|.
%     \item[\docissuedo] |\cal{F}| or |\mathcal{F}|
%     \end{itemize}
%   \item \docfillblank
%     \begin{itemize}
%     \item[\docissuedont] |\[x_0,x\]|
%     \item[\docissuedo] |\left[x_0,x\right]|
%     \end{itemize}
%   \item Also see \autoref{part:other}, \autoref{other:issue}
%   \end{enumerate}
%   
%   \clearpage
%   \part{Listing}\label{part:listing}
%   
%   \doclist{pre} are settings to replicate the listings.
%   For exhaustivity, check the \env{documentation} section of \pkg{ccool}\texttt{.dtx}.
%   
%   \doclist{vers} is self explanatory.
%   
%   \doclist{tut:plain}-\ref{listing:tut:descr:ii} is a tutorial comprising different ways to typeset
%   ``Let $\mathbb{N}$ and $\mathbb{R}$\dots'' The plain version is prone to errors,
%   should the author change |\mathbb{R}| to, say, |\mathcal{R}| throughout the document.
%   \doclist{tut:cmd} avoids that by separating style from meaning.
%   \doclist{tut:ccool}-\ref{listing:tut:s} achieve greater compactness (DRY and expand in place).
%   \doclist{tut:descr:i} extends \doclist{tut:ccool} with a description for each key.
%   \doclist{tut:descr:ii} is to \doclist{tut:descr:i}, what \doclist{tut:s} is \doclist{tut:i}.
%   
%   \doclist{sep} shows the full range of uses of \nameref{usage:par:s}.
%   
%   \doclist{hw:i} and \doclist{hw:ii} are a contrived ``Hello, world!'' test case.
%   
%   \doclist{proba:i}, \doclist{mvt:i}, and \doclist{cusum:i} typeset realistic mathematical text, and write it to a file.
%   \doclist{proba:ii}, \doclist{mvt:ii}, and \doclist{cusum:ii}, read the corresponding files.
%   
%   \newtcblisting[auto counter]
%   {listing}[2][]{
%   noparskip,
%   breakable,
%   colback=white,
%   colframe=black,
%   opacitybacktitle=.8,%
%   fonttitle=\bfseries,
%   title={Listing~\thetcbcounter. #1},
%   arc=0pt,
%   outer arc=0pt,
%   boxrule=1pt,
%   listing and text,
%   #2}
%   
%   \phantomsection\addcontentsline{toc}{section}
%   {\ref{listing:pre}. Preamble}
%   \iffalse
%<*guardlisting>   
%   \fi
\begin{listing}[Preamble]
  {label=listing:pre, listing only}
  \usepackage{amsmath, amsthm, commath}
  \usepackage[T1]{fontenc}% \char`[
\end{listing}
% \iffalse
%</guardlisting> 
% \fi
% \phantomsection\addcontentsline{toc}{section}
% {\ref{listing:vers}.\cs{CcoolVers}}
% \iffalse
%<*guardlisting> 
% \fi
\begin{listing}[\cs{CcoolVers}]
  {label=listing:vers}
  \CcoolVers
\end{listing}
% \iffalse
%</guardlisting> 
% \fi
% 
% \phantomsection\addcontentsline{toc}{section}
% {\ref{listing:tut:plain}. Let $\mathbb{N}$ and $\mathbb{R}$\dots Plain}
% \iffalse
%<*guardlisting> 
% \fi
\begin{listing}[Let $\mathbb{N}$ and $\mathbb{R}$\dots Plain]
  {label=listing:tut:plain, listing and text}
  Let~$\mathbb{N}$ and $\mathbb{R}$ denote the natural and real numbers.
\end{listing}
% \iffalse
%</guardlisting> 
% \fi
% 
% \phantomsection\addcontentsline{toc}{section}
% {\ref{listing:tut:cmd}. Let $\mathbb{N}$ and $\mathbb{R}$\dots Equivalent to \ref{listing:tut:plain} with \cs{NewDocumentCommand}}
% \iffalse
%<*guardlisting> 
% \fi
\begin{listing}[Let $\mathbb{N}$ and $\mathbb{R}$\dots Equivalent to \ref{listing:tut:plain} with \cs{NewDocumentCommand}]
  {label=listing:tut:cmd, listing and text}
  \NewDocumentCommand\Nat{}{\mathbb{N}}
  \NewDocumentCommand\Real{}{\mathbb{R}}
  Let~$\Nat$ and $\Real$ denote the natural and real numbers.
\end{listing}
% \iffalse
%</guardlisting> 
% \fi
% 
% \phantomsection\addcontentsline{toc}{section}
% {\ref{listing:tut:ccool}. Let $\mathbb{N}$ and $\mathbb{R}$\dots Equivalent to \ref{listing:tut:cmd} with \cs{Ccool}}
% \iffalse
%<*guardlisting> 
% \fi
\begin{listing}[Let $\mathbb{N}$ and $\mathbb{R}$\dots Equivalent to \ref{listing:tut:cmd} with \cs{Ccool}]
  {label=listing:tut:ccool, listing and text}
  \Ccool { Nat = {\mathbb{N}}, Real = {\mathbb{R}} }
  Let~$\Nat$ and $\Real$~denote the natural and real numbers.
\end{listing}
% \iffalse
%</guardlisting> 
% \fi
% 
% \phantomsection\addcontentsline{toc}{section}
% {\ref{listing:tut:i}. Let $\mathbb{N}$ and $\mathbb{R}$\dots Equivalent to \ref{listing:tut:ccool} with \docopte{i}{\meta{\pkgparin}}}
% \iffalse
%<*guardlisting> 
% \fi
\begin{listing}[Let $\mathbb{N}$ and $\mathbb{R}$\dots Equivalent to \ref{listing:tut:ccool} but with \docopte{i}{\meta{\pkgparin}}]
  {label=listing:tut:i, listing and text}
  \Ccool i{\mathbb{#1}}{ Nat = {N}, Real = {R} }
  Let~$\Nat$ and $\Real$~denote the natural and real numbers.
\end{listing}
% \iffalse
%</guardlisting> 
% \fi
% 
% \phantomsection\addcontentsline{toc}{section}
% {\ref{listing:tut:s}. Let $\mathbb{N}$ and $\mathbb{R}$\dots Equivalent to \ref{listing:tut:i} with \docopte{s}{\meta{\pkgparsi}}}
% \iffalse
%<*guardlisting> 
\fi
\begin{listing}[Let $\mathbb{N}$ and $\mathbb{R}$\dots Equivalent to \ref{listing:tut:i} with \docopte{s}{\meta{\pkgparsi}}]
  {label=listing:tut:s, listing and text}
  \Ccool[Let~]
  i{\mathbb{#1}}{ Nat = {N}, Real = {R} }*s{{~\rm{and}~}}
  [~denote the natural and real numbers.]{}
\end{listing}
% \iffalse
%</guardlisting> 
% \fi
% 
% \phantomsection\addcontentsline{toc}{section}
% {\ref{listing:tut:descr:i}. Let $\mathbb{N}$ and $\mathbb{R}$\dots Extends \ref{listing:tut:s} with descriptive keys, with \docoptd{\meta{\pkgparpa}}}
% \iffalse
%<*guardlisting> 
% \fi
\begin{listing}[Let $\mathbb{N}$ and $\mathbb{R}$\dots Equivalent to \ref{listing:tut:s} with \docoptd{\meta{\pkgparpa}}]
  {label=listing:tut:descr:i, listing and text}
  \Ccool{ Nat = {\mathbb{N}}, Real = {\mathbb{R}} }[]
  <Describe>{ Nat = {natural numbers}, Real = {the real line} }
  [Let $\Nat$ and $\Real$ denote the~\Nat<Describe>~and~\Real<Describe>.]{}
\end{listing}
% \iffalse
%</guardlisting> 
% \fi
% 
% \phantomsection\addcontentsline{toc}{section}
% {\ref{listing:tut:descr:ii}. Let $\mathbb{N}$ and $\mathbb{R}$\dots Equivalent to \ref{listing:tut:descr:i} with \docopte{s}{\meta{\pkgparsi}}}
% \iffalse
%<*guardlisting> 
% \fi
\begin{listing}[Let $\mathbb{N}$ and $\mathbb{R}$\dots Equivalent to \ref{listing:tut:descr:i} with \docopte{s}{\meta{\pkgparsi}}]
 {label=listing:tut:descr:ii, listing and text}
 \Ccool[Let~]
 i{\mathbb{#1}}{ Nat = {N}, Real = {R} }*s{{~\rm{and}~}}
 [~denote~the~]
 <Describe>
 {
   Nat = {natural numbers},
   Real = {the real line}
 }*s{{~and~}}o{#1.}
\end{listing}
%\iffalse
%</guardlisting> 
% \fi
% 
% \phantomsection\addcontentsline{toc}{section}
% {\ref{listing:sep}. Separators.}
% \iffalse
%<*guardlisting> 
% \fi   ^^A% Separ: spaces betw. inner and outer brackets matter!->
\begin{listing}[Separators]
  {label=listing:sep}
  \CcoolOption{
    Separ={{\ \char`@\ }{\ \%\ }{\ \char`@\ }}}
  \Ccool{ X = x, Y = y }*[\\]
  { X = x, Y = y, Z = z }*[\\]
  { X = x, Y = y }*s{{\ \&\ }}[\\]
  { X = x, Y = y }*s{{\ \&\ }{,\ }}[\\]
  { X = x, Y = y, Z = z }*s{{\ \&\ }}[\\]
  { X = x, Y = y, Z = z }*s{{\ \&\ }{,\ }}[\\]
  { X = x, Y = y, Z = z }*s{{\ \&\ }{,\ }{\ \&\ }}\\
\end{listing}
% \iffalse
%</guardlisting> 
% \fi
% 
% \phantomsection\addcontentsline{toc}{section}
% {\ref{listing:hw:i}. Hello, world! (test case)}
% \iffalse
%<*guardlisting> 
% \fi
\begin{listing}[Hello, world! (test case)]
 {label=listing:hw:i}
 \CcoolOption{Separ = {{}{.}{.}}, Outer = {####1}}
 \CcoolOption{ Write = \BooleanTrue }
 \Ccool
 <Test>{ KeyA = {.}, KeyB = {!}, KeyC = {\%} }[]
 <Test>{ KeyD = {d}, KeyE = {\%} }[]
 <Test>i{\{#1\}}{ KeyF = {H}, KeyG = {e}, KeyH = {l} }*[]
 <Test>{ KeyI = {\%}, KeyJ = {\%}, KeyK = {\%} }[.\{l\}.\{o\}]
 <Test>{ KeyL = {l}, KeyM = {\char`[}, KeyN = {\char`]} }[]
 <Test>{ KeyO = {o}, KeyP = {\%}, KeyQ = {\%} }[{,\ }]
 <Test>{ KeyR = {w}, KeyS = {o}, KeyT = {r} }*
 s{{}{}{}}o{{\char`[}#1}[]
 <Test>{ KeyU = {\%}, KeyV = {\%}, KeyW = {\%} }[]
 <Test>{ KeyX = {\%}, KeyY = {\%}, KeyZ = {\KeyB<Test>} }\nobreak
 \KeyL<Test>\KeyD<Test>\KeyZ<Test>\KeyN<Test>\\
 \CcoolOption{ Write = \BooleanFalse }
\end{listing}
% \iffalse
%</guardlisting> 
% \fi
% 
% \phantomsection\addcontentsline{toc}{section}
% {\ref{listing:hw:ii}. \doclist{hw:i} read from file}
% \iffalse
%<*guardlisting> 
% \fi
\begin{listing}[\doclist{hw:i} read from file]
 {label=listing:hw:ii}
 \CcoolRead
 \KeyF<Test>\KeyA<Test>\nobreak
 \KeyG<Test>\KeyA<Test>\nobreak
 \KeyH<Test>\KeyA<Test>\nobreak
 \KeyH<Test>\KeyA<Test>\nobreak
 {\{}\nobreak\KeyO<Test>{\}},{\ }\nobreak
 \KeyM<Test>\KeyR<Test>\nobreak
 \KeyO<Test>\nobreak
 \KeyT<Test>\nobreak
 \KeyL<Test>\nobreak
 \KeyD<Test>\nobreak
 \KeyZ<Test>\nobreak
 \KeyN<Test>\nobreak
\end{listing}
% \iffalse
%</guardlisting> 
% \fi
% 
% \phantomsection\addcontentsline{toc}{section}{\ref{listing:proba:i}. Probability space.}
% \iffalse
%<*guardlisting> 
% \fi
\begin{listing}[Probability space]
  {label=listing:proba:i}
  \CcoolOption{ Write = \BooleanTrue }
  \Ccool[Let~]
  { Space = \Omega, Field = \mathcal{F}, Meas = \mathcal{P} }
  *s{{,}}o{$\{#1\}$}
  [~denote the probability space, where~]{ PowerSet = { 2^{\Space} } }
  [$\Field\subset \PowerSet$.]
  {}
  \CcoolOption{ Write = \BooleanFalse }
\end{listing}
% \iffalse
%</guardlisting> 
% \fi
% 
% \phantomsection\addcontentsline{toc}{section}{\ref{listing:proba:ii}. \doclist{proba:i} read from file}
% \iffalse
%<*guardlisting> 
% \fi
\begin{listing}[\doclist{proba:i} read from file]
  {label=listing:proba:ii}
  \CcoolRead \tab $\Omega$ $\Field$ $\Meas$
\end{listing}
% \iffalse
%</guardlisting> 
% \fi
% 
% \phantomsection\addcontentsline{toc}{section}{\ref{listing:mvt:i}. Mittelwertsatz f\"ur $n$ Variable.}
% \iffalse
%<*guardlisting> 
% \fi
% \begin{listing}[{Mittelwertsatz f\"ur $n$ Variable\cite[17.3]{tcolbox}}]
%   {label=listing:mvt:i}
%   \CcoolOption{ Write = \BooleanTrue }
%   \newtheorem{theorem}{Theorem}
%   \AfterEndEnvironment{theorem}{\CcoolHook}
%   \Ccool i{\mathbb{#1}}
%   { N = { N } , R = { R } }+[]
%   { Grad = { \operatorname{grad} } }+
%   [\begin{theorem}
%     [Mittelwertsatz f\"ur $n$ Variable]Es~sei~]
%     { OffMenge = {D}, Ci = {C^{1}}, Strecke = { \left[x_0,x\right] } }+
%     [$n\in\N$,~$\OffMenge\subseteq\N^n$ eine offene Menge und $f\in\Ci(\OffMenge,\R)$.
%     Dann gibt es auf jeder Strecke $\Strecke\subset\OffMenge$ einen Punkt $\xi\in\Strecke$,~]
%     { Steig = { \frac{ f(x)-f(x_0) }{ x-x_0 } }, Punkt = { \xi } }+
%     [so dass gilt
%     \begin{equation*}
%       \Steig = \Grad f(\Punkt)^{\top}
%     \end{equation*}
%   \end{theorem}]
%   {}
%   (Check: $\N$, $\Punkt$)
%   \CcoolOption{ Write = \BooleanFalse }
% \end{listing}
% \iffalse
%</guardlisting> 
% \fi
% 
% \phantomsection\addcontentsline{toc}{section}{\ref{listing:mvt:ii}. \doclist{mvt:i} read from file}
% \iffalse
%<*guardlisting> 
% \fi
\begin{listing}[\doclist{mvt:i} read from file]
  {label=listing:mvt:ii}
  \CcoolRead \tab $\N$ $\R$ $\OffMenge$ $\Ci$ $\Strecke$ 
\end{listing}
% \iffalse
%</guardlisting> 
% \fi
% 
% \phantomsection\addcontentsline{toc}{section}{\ref{listing:lambda:i}. Fonction et fonctionelle}
% \iffalse
%<*guardlisting> 
% \fi
\begin{listing}[Fonction et fonctionelle]
  {label=listing:lambda:i}
  \CcoolOption{ Write = \BooleanTrue }
  \Ccool{ EvalAt = \CcoolLambda{(#1)}, ApplyOp = \CcoolLambda[mm]{#1[#2]} }
  [Supposons une fonction $f\EvalAt{t}$, et \'etudions le probl\`eme o\`u la fonctionnelle $\ApplyOp{S}{f}$ est donn\'ee par\dots]{}
  \CcoolOption{ Write = \BooleanFalse }
\end{listing}
% \iffalse
%</guardlisting> 
% \fi
% 
% \phantomsection\addcontentsline{toc}{section}{\ref{listing:lambda:ii}. \doclist{lambda:i} read from file}
% \iffalse
%<*guardlisting> 
% \fi
% \begin{listing}[\doclist{lambda:i} read from file]
%   {label=listing:lambda:ii}
%   \CcoolRead \tab $f\EvalAt{t}$, $\ApplyOp{S}{f}$
% \end{listing}
% \iffalse
%</guardlisting> 
% \fi
% 
% \phantomsection\addcontentsline{toc}{section}{\ref{listing:cusum:i}. CUSUM statistic.}
% \iffalse
%<*guardlisting> 
% \fi
\begin{listing}[CUSUM statistic\cite{ccool-thesis}]
  {label=listing:cusum:i}
  \newtheorem{definition}{Definition}
  \AfterEndEnvironment{definition}{\CcoolHook}
  
  \CcoolOption{ Write = \BooleanTrue }
  \Ccool{ SuchThat = { ;~ }, Time = { t }, Process = { \xi }, StopT = { T }, EvalAt = \CcoolLambda{(#1)}  }
  [The CUSUM statistic process and the corresponding one-sided CUSUM stopping time are defined as follows:
  \begin{definition}\label{the CUSUM statistic}. Let~]
    { Scale = { \lambda }, Real = {\mathcal{R}} }+*s{{~\in~}}[~and~]
    { CUSUMthresh = { \nu } }+*o{$#1\in\Real^{+}$.}
    [~Define the following processes:]
    { LogWald = { u },  CUSUMst = { \StopT_{c} }, CUSUM = { y }, LogWaldInf = { m } }+
    [\begin{enumerate}
    \item{$\LogWald_{\Time}\EvalAt{ \Scale } = \Scale\Process_{\Time} - \frac{1}{2}\Scale^2\Time$;
        $\LogWaldInf_{\Time}\EvalAt{ \Scale } = \inf_{ 0\le s \le \Time }\CUSUM_{s} \EvalAt{ \Scale }$.}
    \item{$\CUSUM_{\Time}\EvalAt{ \Scale } = \LogWaldInf_{\Time}\EvalAt{ \Scale } - \LogWald_{\Time}\EvalAt{ \Scale }\ge0$, which is the CUSUM statistic process.}
    \item{$\CUSUMst \EvalAt{ \Scale, \LogWaldInf } = \inf\left[ \Time \ge 0 \SuchThat \CUSUM_{\Time}\EvalAt{\Scale} \ge \LogWaldInf \right]$, which is the CUSUM stopping time.} \end{enumerate}\end{definition}\par]{}
  
  (Check: $\Scale$, $\CUSUM$)
  \CcoolOption{ Write = \BooleanFalse }
\end{listing}
% \iffalse
%</guardlisting> 
% \fi
% 
% \phantomsection\addcontentsline{toc}{section}{\ref{listing:cusum:ii}. \doclist{cusum:i} read from file}
% \iffalse
%<*guardlisting> 
% \fi
% \begin{listing}[\doclist{cusum:i} read from file]
%   {label=listing:cusum:ii}
%   \CcoolRead \tab $\Time $ $\Process$ $\Scale$ $\Real$ $\CUSUMthresh$ $\LogWald$  $\CUSUMst$ $\CUSUM$ $\LogWaldInf$ 
% \end{listing}
% \iffalse
%</guardlisting> 
% \fi
% 
% \clearpage
% \part{Other}\label{part:other}
% 
% \section{Acknowledgment}\label{other:acknowl} 
% 
% This work has benefited from Q\&A's from the \LaTeX community\cite{user-erw}\cite{p158966}.
% Specific attributions are made throughout this document.
%
% \section{Genealogy}\label{other:geneal}
% In \docvers{2}{0}, the abstract began with
% \begin{quote}
%\pkg{ccool} stands for Custom COntent Oriented for \LaTeX,
%     that is `` give commands the ability to contain the mathematical meaning while retaining the typesetting versatility'',
%     a concept pioneered by \pkg{cool}\cite{cool}
%   \end{quote}
% But through forum interactions\cite{p158966}, the author realized it was confusing and the scope of \pkg{ccool} is in fact more generic.
%
% \section{Install}\label{other:install}
% \begin{enumerate}[label=\emph{\arabic*)}]
% \item Compile \file{ccool.dtx} (under Unix, \texttt{\$tex ccool.dtx})
% \item Put the generated \file{ccool.sty} in the search path of the \LaTeX engine
% \end{enumerate}
% 
% \section{Issue}\label{other:issue}
% 
% \begin{enumerate}[label=\emph{\arabic*)}]
% \item
%   \begin{description}
%   \item[\docissuedont] |Inner=\{####1\}|
%   \item[\docissuesymp] \cs{CcoolRead} fails
%   \item[\docissuedo] |Inner={\char`{####1\char`}}|
%   \end{description}
% \end{enumerate}
% 
% \section{Support}\label{other:support}
% 
% This package is available from \url{https://www.ctan.org/pkg/ccool} and \url{https://github.com/rogard/ccool}.
% 
% \section{Testing}\label{other:testing}
% 
% \subsection{Technicality}
% Not possible to compile-check the expansion of a certain class of macros against predefined values\cite{a-534100}. 
% Instead, one can visually check \autoref{part:listing}, as generated in \autoref{other:install} on one's own machine,
% against that \href{https://github.com/rogard/ccool}{of the repository} for the same version.
% 
% 
% \subsection{Platform}
% \begin{enumerate}[label=\emph{\roman*)}]
% \item 
%   ^^A uname -a
%   \begin{Verbatim}[breaklines=true]
%     Linux laptop 4.15.0-20-generic #21-Ubuntu SMP Tue Apr 24 06:16:15 UTC 2018 x86_64 x86_64 x86_64 GNU/Linux
%   \end{Verbatim}
%   \label{plat:lin}
% \end{enumerate}
% 
% \subsection{Engine}
% \begin{enumerate}[label=\emph{\alph*})]
% \item 
%   \begin{Verbatim}[breaklines=true]
%     pdfTeX 3.14159265-2.6-1.40.20 (TeX Live 2019)
%   \end{Verbatim}
%   \label{eng:tlxviiii:pdf}
% \item 
%   \begin{Verbatim}[breaklines=true]
%     pdfTeX 3.14159265-2.6-1.40.21 (TeX Live 2020)
%   \end{Verbatim}
%   \label{eng:tlxx:pdf}
% \item
%   \begin{Verbatim}[breaklines=true]
%     LuaHBTeX, Version 1.12.0 (TeX Live 2020)
%   \end{Verbatim}
%   \label{eng:tlxx:lua}
% \item
%   \begin{Verbatim}[breaklines=true]
%     XeTeX 3.14159265-2.6-0.999992 (TeX Live 2020)
%   \end{Verbatim}
%   \label{eng:tlxx:xe}
% \end{enumerate}
% 
% \subsection{Results}
% 
% \begin{enumerate}[label=\emph{\arabic*)}]
% \item \pkg{ccool} \docvers{1}{8} satisfactory on platform \ref{plat:lin} and engine \ref{eng:tlxviiii:pdf}
% \item \pkg{ccool} \docvers{1}{8} satisfactory on platform \ref{plat:lin} and engine \ref{eng:tlxx:pdf}
% \item \pkg{ccool} \docvers{1}{9} satisfactory on platform \ref{plat:lin} and engines \ref{eng:tlxx:pdf} and \ref{eng:tlxx:lua}
% \item \pkg{ccool} \docvers{2}{0} satisfactory on platform \ref{plat:lin} and engines \ref{eng:tlxx:pdf},  \ref{eng:tlxx:lua}, and \ref{eng:tlxx:xe}
% \item \pkg{ccool} \docvers{2}{1} satisfactory on platform \ref{plat:lin} and engines \ref{eng:tlxx:pdf},  \ref{eng:tlxx:lua}, and \ref{eng:tlxx:xe}
% \end{enumerate}
% 
% \subsection{Other}
% Check \cite{ccool-thesis} for testing \pkg{ccool} with \cls{llncs}
% 
% \leavevmode
% \refstepcounter{section}
% \docsetnameref{References}
% \label{other:bib}
% \phantomsection\addcontentsline{toc}{section}{References}
% \begin{thebibliography}{1}
% \bibitem{cool} Nick Setzer {\em The \pkg{cool} package}, 2005, \url{https://www.ctan.org/pkg/cool}
% \bibitem{interface3} The \LaTeX3 Project Team {\em The \LaTeX3 interfaces}, 2019,
%   \url{http://ftp.math.purdue.edu/mirrors/ctan.org/macros/latex/contrib/l3kernel/interface3.pdf}
% \bibitem{tcolbox} Thomas F. Sturm {\em The \pkg{tcolorbox} package}, 2019,
%   \url{http://www.texdoc.net/texmf-dist/doc/latex/tcolorbox/tcolorbox.pdf}
% \bibitem{xparse} The \LaTeX3 Project Team {\em The \pkg{xparse} package}, 2020,
%   \url{http://ftp.math.purdue.edu/mirrors/ctan.org/macros/latex/contrib/l3packages/xparse.pdf}
% \bibitem{ccool-thesis} Erwann Rogard and Olympia Hadjiliadis {\em Typesetting a math thesis with \pkg{ccool}}, 2020,
%   \url{https://github.com/rogard/ccool/blob/master/thesis.pdf}
% \bibitem{user-erw} \url{https://tex.stackexchange.com/users/112708/erwann?tab=questions}
% \bibitem{a-188053} \href{https://tex.stackexchange.com/users/17423/sean-allred}{@{}sean-allred}'s answer to
%   ``How to create lambda expressions?'', \url{https://tex.stackexchange.com/a/188053/112708}
% \bibitem{a-534100} \href{https://tex.stackexchange.com/users/73/joseph-wright}{@{}joseph-wright}'s answer to
%   ``Checking a function's expansion against a string'', \url{https://tex.stackexchange.com/a/534100}
% \bibitem{a-536597} \href{https://tex.stackexchange.com/users/73317/frougon}{@{}frougon}'s answer to ``Journaling calls to a function \textdel{taking inline code as argument}'', \url{https://tex.stackexchange.com/a/536620}
% \bibitem{p158966}\cs{Ccool}, extension à \LaTeX à vocation mathématique, \url{http://forum.mathematex.net/latex-f6/ccool-extension-latex-a-vocation-mathematique-t17314.html#p158966}
% \end{thebibliography}
% 
% \changes{\docvers{1}{0}}{2020/03/08}{Initial version}
% \changes{\docvers{1}{1}}{2020/04/04}{Revamped: much of the implementation}
% \changes{\docvers{1}{1}}{2020/04/04}{Replaced: \Arg{kvl_{2}}~by~\docoptd{kvl_{2}}~given that option type \texttt{G} not recommended\cite{xparse}}
% \changes{\docvers{1}{1}}{2020/04/04}{Replaced: \cs{OopsOptions}~by~\cs{OopsOption}}
% \changes{\docvers{1}{1}}{2020/04/04}{Replaced: \docarg{GenericObject}~by~\docarg{Name}}
% \changes{\docvers{1}{1}}{2020/04/04}{Replaced: \docarg{Separators}~by~\docarg{Separ}}
% \changes{\docvers{1}{1}}{2020/04/04}{Added:\cs{OopsTest}}
% \changes{\docvers{1}{1}}{2020/04/04}{Added:\cs{OopsRestore}}
% \changes{\docvers{1}{1}}{2020/04/04}{Added: \docarg{Save}}
% \changes{\docvers{1}{1}}{2020/04/04}{Deleted: Listing 1-5 from \docvers{1}{0}}
% \changes{\docvers{1}{1}}{2020/04/04}{Added: Listing~1., 2., 3., 4., 6., and 9.}
% \changes{\docvers{1}{1}}{2020/04/04}{Fixed: apparent anomaly in \docvers{1}{0}'s Listing~4, see \doclist{hw:i}}
% \changes{\docvers{1}{2}}{2020/04/06}{Added: optional \pkgparex to \cs{OopsNew} as instruction to expand \pkgparde}
% \changes{\docvers{1}{2}}{2020/04/06}{Replaced: \cs{OopsClear}\Arg{\pkgparpa} by \cs{OopsClear}\docopto{\meta{\doccceptkvl}}}
% \changes{\docvers{1}{2}}{2020/04/06}{Deleted: \meta{kvl_{2}}~and~\meta{code_{2}}}
% \changes{\docvers{1}{2}}{2020/04/06}{Deleted: \cs{OopsTest}}
% \changes{\docvers{1}{2}}{2020/04/06}{Deleted: Listing~2-3 from \docvers{1}{1}.}
% \changes{\docvers{1}{2}}{2020/04/06}{Replaced: \cs{Save} by \cs{Write}}
% \changes{\docvers{1}{2}}{2020/04/06}{Replaced: \cs{Restore} by \cs{Read}}
% \changes{\docvers{1}{3}}{2020/04/06}{Replaced: \cs{OopsNew} by \cs{Oops}}
% \changes{\docvers{1}{3}}{2020/04/06}{Replaced: \Arg{\pkgparpa} and \docopto{\meta{\pkgparpa}} by \docoptd{\meta{\pkgparpa}} }
% \changes{\docvers{1}{4}}{2020/04/10}{Replaced: \docopte{s}{\Arg{\pkgparsi}\Arg{\pkgparsii}\Arg{\pkgparsiii}}
% by \docopte{s}{\Arg{\pkgparsi}\docpipe\Arg{\pkgparsi}\Arg{\pkgparsii}\docpipe\Arg{\pkgparsi}\Arg{\pkgparsii}\Arg{\pkgparsiii}}}
% \changes{\docvers{1}{4}}{2020/04/10}{Added:optional \pkgparap to \cs{OopsNew} to make side effects presist beyond local group}
% \changes{\docvers{1}{4}}{2020/04/10}{Added: \cs{OopsHook}}
% \changes{\docvers{1}{4}}{2020/04/10}{Deleted: Listing~1., and 2. }
% \changes{\docvers{1}{4}}{2020/04/10}{Added: Listing~1., 2., and 3. }
% \changes{\docvers{1}{4}}{2020/04/10}{Added: \docarg{Expans} (for debugging' sake, but...) }
% \changes{\docvers{1}{4}}{2020/04/10}{Added: \autoref{usage:dosdont} }
% \changes{\docvers{1}{4}}{2020/04/10}{Added: \cs{OopsDebug} }
% \changes{\docvers{1}{5}}{2020/04/10}{Deleted: dependence on \pkg{datetime} }
% \changes{\docvers{1}{5}}{2020/04/10}{Added: \pkgoptfi }
% \changes{\docvers{1}{6}}{2020/04/10}{Renamed: \pkg{oops} to \pkg{ccool} (better describes the purpose) }
% \changes{\docvers{1}{6}}{2020/04/10}{Renamed: \cs{Oops} to \cs{Ccool}}
% \changes{\docvers{1}{6}}{2020/04/10}{Renamed: \cs{OopsClear} to \cs{CcoolClear}}
% \changes{\docvers{1}{6}}{2020/04/10}{Renamed: \cs{OopsDebug} to \cs{CcoolDebug}}
% \changes{\docvers{1}{6}}{2020/04/10}{Renamed: \cs{OopsHook} to \cs{CcoolHook}}
% \changes{\docvers{1}{6}}{2020/04/10}{Renamed: \cs{OopsOption} to \cs{CcoolOption}}
% \changes{\docvers{1}{6}}{2020/04/10}{Renamed: \cs{OopsRead} to \cs{CcoolRead}}
% \changes{\docvers{1}{6}}{2020/04/10}{Added: \doclist{pre} (preamble) }
% \changes{\docvers{1}{7}}{2020/04/11}{Added: \doclist{cusum:i} (CUSUM) }
% \changes{\docvers{1}{7}}{2020/04/11}{Added: Legends to listings }
% \changes{\docvers{1}{7}}{2020/04/11}{Deleted: \cs{CcoolDebug} }
% \changes{\docvers{1}{7}}{2020/04/11}{Deleted: Listing~5 from \docvers{1}{6} }
% \changes{\docvers{1}{8}}{2020/04/12}{Added: \hyperref[usage:cs:lambda]{\cs{CcoolLambda}} }
% \changes{\docvers{1}{8}}{2020/04/12}{Added: \doclist{lambda:i}, \doclist{lambda:ii} }
% \changes{\docvers{1}{8}}{2020/04/12}{Added: \cs{CcoolVers} }
% \changes{\docvers{1}{8}}{2020/04/12}{Added: \doclist{vers} }
% \changes{\docvers{1}{9}}{2020/04/14}{Added: support for \LuaTeX }
% \changes{\docvers{1}{9}}{2020/04/14}{Moved: from \autoref{part:usage} to \autoref{part:impl}, what is now that part's \autoref{impl:frontend}}
% \changes{\docvers{2}{0}}{2020/04/15}{Deleted: \pkgoptfi's dependency on \pkg{texosquery} and \cs{pdfcreationdate}}
% \changes{\docvers{2}{0}}{2020/04/15}{Added: support for \XeTeX}
% \changes{\docvers{2}{0}}{2020/04/15}{Updated: \cs{RequirePackage}, \cs{NeedsTeXFormat}'s second argument / TeX Live 2020}
% \changes{\docvers{2}{1}}{2020/04/17}
% {Replaced: \hyperref[usage:cs:lambda]{\cs{CcoolLambda}}'s optional integer argument (number of \texttt{m}'s)
% by a standard argument list}
% \changes{\docvers{2}{1}}{2020/04/17}{Replaced: \hyperref[usage:par:pa]{\meta{\pkgparpa}}'s position within \cs{Ccool}'s argument list, from first to second. Greater versatility }
% \changes{\docvers{2}{1}}{2020/04/17}{Replaced: as~the~default~of~\pkgoptpa,~\docarg{Math}~by~\pkgoptpad}
% \changes{\docvers{2}{1}}{2020/04/17}{Replaced: \docarg{Name} by \nameref{usage:opt:pa} }
% \changes{\docvers{2}{1}}{2020/04/17}
% {Added:
% \doclist{tut:plain},
% \doclist{tut:cmd},
% \doclist{tut:ccool},
% \doclist{tut:i},
% \doclist{tut:i},
% \doclist{tut:descr:i},
% and \doclist{tut:descr:ii} (tutorial)
% }
% \changes{\docvers{2}{2}}{2020/04/20}{Removed: \% from listings}
% \changes{\docvers{2}{2}}{2020/04/20}{Replaced: part of the abstract's with more straighforward descriptions based on input from forum participtants  }
%   
%   \PrintChanges
%   \PrintIndex
%   \clearpage
%   \StopEventually{
%   ^^A   \PrintChanges
%   ^^A   \PrintIndex
% }
% \end{documentation}
% ^^A% Commented out to eliminate WARNING: Reference `doc/function//
% \begin{implementation}
%   \part{Implementation}\label{part:impl}
%   
%   \iffalse
%<*package>   
%   \fi
%   \section{Opening}
%    \begin{macrocode}
%<@@=ccool>      
\ExplSyntaxOn
%    \end{macrocode}
% \section{\texttt{aux}}
% \begin{macro}{\@@_aux_inner_set:n}
%   \begin{arguments}
%   \item \meta{code}
%   \end{arguments}
%    \begin{macrocode}
\cs_new_protected:Nn \@@_aux_inner_set:n
{
  \cs_gset:Npn \@@_aux_inner:n ##1 {#1}
  \cs_generate_variant:Nn \@@_aux_inner:n { e }
}
%    \end{macrocode}
% \end{macro}
% \begin{macro}{\@@_aux_key:w }
%   \begin{arguments}
%   \item \meta{ key }
%   \item \meta{ value }
%   \end{arguments}
%    \begin{macrocode}
\cs_new_protected:Npn \@@_aux_key:w #1 = #2 \q_stop
{
  \seq_gput_right:Nx \g@@_aux_key_seq { \tl_trim_spaces:n{#1} }
}
%    \end{macrocode}
% \end{macro}
% \begin{macro}{\@@_aux_key:n }
%   \begin{arguments}
%   \item \meta{ key = value }
%   \end{arguments}
%    \begin{macrocode}
\cs_new_protected:Nn \@@_aux_key:n
{
  \@@_aux_key:w #1 \q_stop
}
%    \end{macrocode}
% \end{macro}
% \begin{macro}{\@@_aux_key:N }
%   \begin{arguments}
%   \item \meta{ seq }
%   \end{arguments}
%    \begin{macrocode}
\cs_new_protected:Nn \@@_aux_key:N 
{
  \seq_gclear_new:N \g@@_aux_key_seq
  \seq_map_function:NN #1 \@@_aux_key:n
}
%    \end{macrocode}
% \end{macro}
% \begin{macro}{\@@_aux_outer_set:n}
%   \begin{arguments}
%   \item \meta{ inline code }
%   \end{arguments}
%    \begin{macrocode}
\cs_new_protected:Nn \@@_aux_outer_set:n
{
  \cs_gset:Npn \@@_aux_outer:n ##1 {#1}
}
%    \end{macrocode}
% \end{macro}
% \begin{macro}{\@@_aux_prop:nn}
%    \begin{macrocode}
\prop_new:N \g@@_aux_prop
\cs_new_protected:Nn \@@_aux_prop:nn 
{
  \prop_gput:Nnn \g@@_aux_prop{#1}{#2}
}
\cs_generate_variant:Nn \@@_aux_prop:nn { eo, ee, ex, xo, xe, xx }
%    \end{macrocode}
% \end{macro}
% \begin{macro}{\@@_aux_prop:w}
%   \begin{arguments}
%   \item \meta{ key }
%   \item \meta{ value }
%   \end{arguments}
%    \begin{macrocode}
\tl_new:N \g@@_option_expans_tl
\cs_new_protected:Npn \@@_aux_prop:w #1 = #2 \q_stop
{
  \exp_args:Nx
  \use:c{@@_aux_prop:\g@@_option_expans_tl}
  { \tl_trim_spaces:n{#1} }
  { \@@_aux_inner:n{ \tl_trim_spaces:n{#2} } }
}
%    \end{macrocode}
% \end{macro}
% \begin{macro}{\@@_aux_prop:n}
%   \begin{arguments}
%   \item \meta{ key = value }
%   \end{arguments}
%    \begin{macrocode}
\cs_new_protected:Nn \@@_aux_prop:n
{
  \@@_aux_prop:w #1 \q_stop 
}
%    \end{macrocode}
% \end{macro}
% \begin{macro}{\@@_aux_prop:N}
%   \begin{arguments}
%   \item \meta{keyval list}
%   \end{arguments}
%    \begin{macrocode}
\cs_new_protected:Nn \@@_aux_prop:N
{
  \prop_gclear_new:N \g@@_aux_prop
  \seq_if_empty:NTF #1
  { \c_empty_tl }
  {
    \seq_map_function:NN #1 \@@_aux_prop:n
  }
}
%    \end{macrocode}
% \end{macro}
% \begin{macro}{\@@_aux_separ:nn}
%   \begin{arguments}
%   \item \meta{ int }
%   \item \meta{ tokens }
%   \end{arguments}
%    \begin{macrocode}
\cs_new:Nn \@@_aux_separ:nn
{
  \int_case:nnTF {#1}
  {
    {1}
    { \prg_replicate:nn{ 3 }{#2} }
    {2}
    {
      { \use_i:nn #2 }
      { \use_ii:nn #2 }
      { \use_i:nn #2 }
    }
    {3}{#2}
  }
  { \c_empty_tl }
  {
    \msg_error:nnnn { @@ }
    { separ }
    { \exp_not:N \_@@_aux_separ:nn }
    {#2}
  }
}
\cs_generate_variant:Nn \@@_aux_separ:nn { e }
%    \end{macrocode}
% \end{macro}
% \begin{macro}{\@@_aux_separ:n}
%   \begin{arguments}
%   \item \meta{ tokens }
%   \end{arguments}
%    \begin{macrocode}
\cs_new:Nn \@@_aux_separ:n
{
  \@@_aux_separ:en{ \tl_count:n{#1} }{#1}
}
%    \end{macrocode}
% \end{macro}
% \begin{macro}{\@@_aux_val:Nn}
%   \begin{arguments}
%   \item \meta{ seq }
%   \item \meta{ tl var name }
%   \end{arguments}
%    \begin{macrocode}
\cs_new_protected:Nn \@@_aux_val:Nn
{
  \seq_gclear_new:N \g@@_aux_val_seq
  \@@_seq_from_prop:NNn \g@@_aux_val_seq #1 { \@@_prop_name:n{#2} } 
}
%    \end{macrocode}
% \end{macro}
% \section{\texttt{lambda}}
% \begin{macro}{\@@_lambda:nn}\cite{a-188053}
%    \begin{macrocode}
\cs_new_protected:Npn \@@_lambda:nn #1 #2 
{
  \exp_args:NNx 
  \DeclareDocumentCommand \@@_lambda_expression 
  {#1}
  {#2}
  \@@_lambda_expression
}
%    \end{macrocode}
% \end{macro}
% \section{\texttt{log}}
% \begin{macro}{\@@_log_close:}
%    \begin{macrocode}
\iow_new:N \g@@_log_iow
\AtEndDocument{\iow_close:N \g@@_log_iow}
\bool_set_false:N \g@@_log_open_bool
\cs_new_protected:Nn \@@_log_close:
{
  \iow_close:N \g@@_log_iow
  \bool_gset_false:N \g@@_log_open_bool
}
%    \end{macrocode}
% \end{macro}
% \begin{macro}{\@@_log_open:}
%    \begin{macrocode}
\tl_new:N \g@@_log_file_tl
\cs_new_protected:Nn \@@_log_open:
{
  \tl_gset:Nx \g@@_log_to_tl{\g@@_log_file_tl}
  \iow_open:Nn \g@@_log_iow {\g@@_log_to_tl}
  \bool_gset_true:N \g@@_log_open_bool
}
%    \end{macrocode}
% \end{macro}
% \begin{macro}{\@@_log_read:n}
%   \begin{arguments}
%   \item \meta{path}
%   \end{arguments}
%    \begin{macrocode}
\cs_new_protected:Nn \@@_log_read:n
{
  \file_input:n{#1}
  \tl_log:n{read~from~#1}
}
\cs_generate_variant:Nn \@@_log_read:n { e }
%    \end{macrocode}
% \end{macro}
% \begin{macro}{\@@_log_read:}
%    \begin{macrocode}
\cs_new_protected:Nn \@@_log_read:
{
  \@@_log_read:e{\g@@_log_to_tl}
}
%    \end{macrocode}
% \end{macro}
% \begin{macro}{\@@_log_write:n}
%    \begin{macrocode}
\tl_new:N \g@@_log_to_tl
\cs_new_protected:Nn \@@_log_write:n
{
  \bool_if:nTF{ \g@@_log_open_bool }
  {
    \iow_now:Nn \g@@_log_iow {#1}
    \tl_log:n{ write~to~#1 }
  }
  { \msg_error:nnn{ @@ }{ iow }{ \g@@_log_iow }  }
}
\cs_generate_variant:Nn \@@_log_write:n { e }
%    \end{macrocode}
% \end{macro}
% \section{\texttt{make_key}}
% \begin{macro}{\@@_make_key:Nn}
%   \begin{arguments}
%   \item \meta{ token }
%   \item \meta{ key }
%   \end{arguments}
%    \begin{macrocode}
\cs_new_protected:Nn \@@_make_key:Nn 
{
  \exp_args:NNx
  \ProvideDocumentCommand{#1} 
  { D<>{\g@@_option_param_tl} }
  {
    \@@_prop_item:nn{##1}{#2}
  }
}
\cs_generate_variant:Nn \@@_make_key:Nn {c}
%    \end{macrocode}
% \end{macro}
% \begin{macro}{\@@_make_key:n}
%   \begin{arguments}
%   \item \meta{ key }
%   \end{arguments}
%    \begin{macrocode}
\cs_new_protected:Nn \@@_make_key:n
{
  \@@_make_key:cn{#1}{#1}
}
\cs_generate_variant:Nn \@@_make_key:n { e }
%    \end{macrocode}
% \end{macro}
% \begin{macro}{\@@_make_key:N}
%   \begin{arguments}
%   \item \meta{ seq }
%   \end{arguments}
%    \begin{macrocode}
\cs_new_protected:Nn \@@_make_key:N
{
  \seq_map_function:NN #1 \@@_make_key:e
}
%    \end{macrocode}
% \end{macro}
% \section{\texttt{make_ccool}}
% \begin{macro}{\@@_make_ccool_exp:nnn}
%    \begin{macrocode}
\cs_new_protected:Nn \@@_make_ccool_exp:nnn
{
  \@@_aux_val:Nn \g@@_aux_key_seq {#1}
  \@@_aux_outer_set:n{#3}
  \@@_aux_outer:n
  {
    \exp_args:NNf
    \@@_seq_use:Nn
    \g@@_aux_val_seq
    {#2}
  }
}
%    \end{macrocode}
% \end{macro}
% \begin{macro}{\@@_make_ccool_key:nnn}
%    \begin{macrocode}
\cs_new_protected:Nn \@@_make_ccool_key:nnn
{
  \@@_prop_if_exist:nTF{#1}
  { \c_empty_tl }
  { \@@_prop_new:n{#1} }
  \exp_args:No \@@_aux_inner_set:n{#2}
  \seq_set_from_clist:Nn \g@@_aux_keyval_seq {#3}
  \@@_aux_prop:N \g@@_aux_keyval_seq
  \@@_prop_append:Nn \g@@_aux_prop {#1}
  \@@_aux_key:N \g@@_aux_keyval_seq
  \@@_make_key:N \g@@_aux_key_seq
}
%    \end{macrocode}
% \end{macro}
% \begin{macro}{\@@_make_ccool_sideeffect:nnn}\cite{a-536597}
%    \begin{macrocode}
\cs_new_protected:Nn \@@_make_ccool_sideeffect:nnn
{
  \@@_make_ccool_key:nnn{#1}{#2}{#3}
  \bool_if:nTF{ \g@@_log_open_bool }
  {
    \@@_log_write:n
    {
      \begingroup
      \def \@@_log_entry { \Ccool<#1>i{#2}{#3} } \expandafter
      \endgroup \@@_log_entry      
    }
  }{\c_empty_tl}
}
%    \end{macrocode}
% \end{macro}
% \begin{macro}{\@@_make_ccool:nnnn}
%   \begin{arguments}
%   \item \meta{ token list }
%   \item \meta{ seq_{1} }
%   \item \meta{ seq_{2} } 
%   \item \meta{ prop }
%   \end{arguments}
%    \begin{macrocode}
\cs_new_protected:Npn \@@_make_ccool:nnnn #1 #2 #3 #4
{ 
  \exp_args:NNx \DeclareDocumentCommand \Ccool
  {%^^A      2    3         4 5  6    7 8              9
    +o D<>{#1} E{ i }{{#2}} m t+ s E{ s o }{{#3}{#4}} +o
  }
  {
    \IfValueT{##1}{##1}    
    \@@_make_ccool_sideeffect:nnn{##2}{##3}{##4}    
    \IfBooleanT{##6}
    {
      \@@_make_ccool_exp:nnn{##2}{##7}{##8}
    }
    \bool_if:nTF{##5}
    {
      \gappto{\CcoolHook}
      {
        \@@_make_ccool_sideeffect:nnn{##2}{##3}{##4}
      }
    }
    {\c_empty_tl}
    \IfValueT{##9}
    {
      \exp_not:n{ \Ccool[##9] }
    }
  }
}
%    \end{macrocode}
% \end{macro}
% \section{\texttt{msg}}
%    \begin{macrocode}
\msg_new:nnn {@@}{ generic }{#1}
\msg_new:nnn {@@}{ iow }{#1~is~closed~can't~write}
\msg_new:nnn {@@}{ keyonly }{#1~does~not~take~values;~keyval~is~#2}
\msg_new:nnn {@@}{ keywrong }{#1~does~not~recognize~key~#2}
\msg_new:nnn {@@}{ separ }{#1~expects~1~to~3~items,~#2}
\msg_new:nnn {@@}{ unset }{#1~unset}
%    \end{macrocode}
% \section{\texttt{option}}
% \begin{macro}{\@@_option_inner:n}
%   \begin{arguments}
%   \item \meta{code}
%   \end{arguments}
%    \begin{macrocode}
\cs_new_protected:Nn \@@_option_inner:n
{
  \tl_gset:Nn \g@@_option_inner_tl {#1}
}
\@@_option_inner:n
{
  \msg_warning:nnn{ @@ }{ unset }{ \exp_not:N \g@@_option_inner_tl }
}
%    \end{macrocode}
% \end{macro}
% \begin{macro}{\@@_option_param:n}
%   \begin{arguments}
%   \item \meta{token list}
%   \end{arguments}
%    \begin{macrocode}
\cs_new:Nn \@@_option_param:n
{
  \tl_gset:Nn \g@@_option_param_tl{#1}
}
\@@_option_param:n
{  
  \msg_error:nnx{ @@ }
  { generic }
  { \exp_not:N\g@@_option_param_tl~undefined }
}
%    \end{macrocode}
% \end{macro}
% \begin{macro}{\@@_option_outer:n}
%   \begin{arguments}
%   \item \meta{ inline code }
%   \end{arguments}
%    \begin{macrocode}
\cs_new_protected:Nn \@@_option_outer:n
{
  \tl_gset:Nn \g@@_option_outer_tl {#1}
}
\@@_option_outer:n
{
  \msg_warning:nnn{ @@ }{ unset }{ \exp_not:N \g@@_option_outer_tl }
}
%    \end{macrocode}
% \end{macro}
% \begin{macro}{\@@_option_separ:n}
%   \begin{arguments}
%   \item \Arg{ tl_{1} }\Arg{ tl_{2} }\Arg{ tl_{3} }
%   \end{arguments}
%    \begin{macrocode}
\cs_new_protected:Nn \@@_option_separ:n
{
  \cs_gset:Npn \g@@_option_separ_tl {#1}
}
\@@_option_separ:n
{
  \msg_warning:nnn{ @@ }{ unset }{ \exp_not:N \g@@_option_separ_tl }
}
%    \end{macrocode}
% \end{macro}
% \section{\texttt{prop}}
% \begin{macro}{\@@_prop_append:NN}
%   \begin{arguments}
%   \item \meta{ prop_{1} }
%   \item \meta{ prop_{2} }
%   \end{arguments}
%    \begin{macrocode}
\cs_new_protected:Npn \@@_prop_append:NN #1 #2
{
  \cs_set:Nn \@@_prop_append:nn
  {
    \prop_gput:Nnx #1 {##1}{ \prop_item:Nn #2{##1} }
  }
  \prop_map_function:NN #2 \@@_prop_append:nn
}
\cs_generate_variant:Nn \@@_prop_append:NN { cN }
%    \end{macrocode}
% \end{macro}
% \begin{macro}{\@@_prop_append:Nn}
%   \begin{arguments}
%   \item \meta{ prop }
%   \item \meta{ tl var name }
%   \end{arguments}
%    \begin{macrocode}
\cs_new_protected:Nn \@@_prop_append:Nn 
{
  \@@_prop_append:cN{ \@@_prop_name:n {#2} } #1
}
%    \end{macrocode}
% \end{macro}
% \begin{macro}{\@@_prop_clear_new:n}
%   \begin{arguments}
%   \item \meta{ tl var name }
%   \end{arguments}
%    \begin{macrocode}
\cs_new_protected:Nn \@@_prop_clear_new:n
{
  \exp_args:No \prop_clear_new:c{ \@@_prop_name:n {#1} }
}
%    \end{macrocode}
% \end{macro}
% \begin{macro}{\@@_prop_clear_new_map:n}
%   \begin{arguments}
%   \item \meta{ keyval list }
%   \end{arguments}
%    \begin{macrocode}
\cs_new_protected:Nn \@@_prop_clear_new_map:n
{
  \seq_set_from_clist:Nn \g@@_aux_key_seq {#1}
  \seq_map_function:NN \g@@_aux_key_seq \@@_prop_clear_new:n
}
%    \end{macrocode}
% \end{macro}
% \begin{macro}{\@@_prop_if_exist:nTF}
%   \begin{arguments}
%   \item \meta{tl_{1}}
%   \item \meta{tl_{2}}
%   \item \meta{tl_{3}}
%   \end{arguments}
%    \begin{macrocode}
\cs_new:Nn \@@_prop_if_exist:nTF 
{
  \prop_if_exist:cTF{ \@@_prop_name:n {#1} }{#2}{#3}
}
%    \end{macrocode}
% \end{macro}
% \begin{macro}{\@@_prop_item:nn}
%   \begin{arguments}
%   \item \meta{ tl var name }
%   \item \meta{ key }
%   \end{arguments}
%    \begin{macrocode}
\cs_new:Nn \@@_prop_item:nn
{
  \prop_item:cn { \@@_prop_name:n {#1} } {#2}
}
%    \end{macrocode}
% \end{macro}
% \begin{macro}{\@@_prop_name:n}
%   \begin{arguments}
%   \item \meta{ tl var name }
%   \end{arguments}
%    \begin{macrocode}
\cs_new:Npn \@@_prop_name:n #1{ @@_#1 }
%    \end{macrocode}
% \end{macro}
% \begin{macro}{\@@_prop_new:n}
%   \begin{arguments}
%   \item \meta{ tl var name }
%   \end{arguments}
%    \begin{macrocode}
\cs_new_protected:Nn \@@_prop_new:n 
{
  \prop_new:c{ \@@_prop_name:n {#1} }
}
%    \end{macrocode}
% \end{macro}
% \section{\texttt{seq}}
% \begin{macro}{\@@_seq_from_prop:NNn}
%   \begin{arguments}
%   \item \meta{ seq_{1} }
%   \item \meta{ seq_{2} } (keys)
%   \item \meta{ prop }
%   \end{arguments}
%    \begin{macrocode}
\cs_new_protected:Nn \@@_seq_from_prop:NNn
{
  \cs_set_protected:Nn \@@_seq_from_prop:n
  {
    \seq_gput_right:No #1 { \prop_item:cn{#3}{##1} }
  }
  \seq_map_function:NN #2 \@@_seq_from_prop:n
}
%    \end{macrocode}
% \end{macro}
% \begin{macro}{\@@_erw_seq_use:Nn}
%    \begin{macrocode}
%      \begin{arguments}
%      \item \meta{ seq }
%      \item \meta{ tokens }
%      \end{arguments}
\cs_new:Nn \@@_seq_use:Nn
{
  \exp_last_unbraced:NNf
  \seq_use:Nnnn #1
  \@@_aux_separ:n{#2}
}
%    \end{macrocode}
% \end{macro}
% \section{sys}\label{impl:sys}
% \begin{macro}{\@@_sys_date:}
%    \begin{macrocode}
\cs_new:Nn \@@_sys_date:
{
  \int_eval:n
  {
    \c_sys_year_int * 10000
    +\c_sys_month_int * 100
    +\c_sys_day_int *  1
  }
}
%    \end{macrocode}
% \end{macro}
% \begin{macro}{\@@_sys_date_hex:}
%    \begin{macrocode}
\cs_new:Nn \@@_sys_date_hex:
{\int_to_hex:n{\@@_sys_date:}}
%    \end{macrocode}
% \end{macro}
% \begin{macro}{\@@_sys_time:}
%    \begin{macrocode}
\cs_new:Nn \@@_sys_time:
{
  \int_eval:n
  {
    \c_sys_hour_int * 100 
    +\c_sys_minute_int * 1 
  }
}
%    \end{macrocode}
% \end{macro}
% \begin{macro}{\@@_sys_time_hex:}
%    \begin{macrocode}
\cs_new:Nn\@@_sys_time_hex:
{\int_to_hex:n{\@@_sys_time:}}
%    \end{macrocode}
% \end{macro}
% \begin{macro}{\@@_sys_filename:}
%    \begin{macrocode}
\cs_new:Nn\@@_sys_filename:
{
  \c_sys_jobname_str--
  \@@_sys_date_hex:--
  \@@_sys_time_hex:
}
%    \end{macrocode}
% \end{macro}
% \section{Front-end}\label{impl:frontend}
% \leavevmode
% \refstepcounter{subsection}
% \label{usage:cs:clear}
% \docsetnameref{\cs{CcoolClear}}
% \addcontentsline{toc}{subsection}{\protect\numberline{\thesubsection}\cs{CcoolClear}}
% \begin{function}{\CcoolClear}
%   \begin{arguments}
%   \item \meta{\docccepttl}
%   \end{arguments}
%   \begin{description}
%   \item[Semantics] Clears any data created by \nameref{usage:cs:ccool}\Arg{\docccepttl}
%   \end{description}
%    \begin{macrocode}
\NewDocumentCommand{ \CcoolClear }
{ D<>{\g@@_option_param_tl} }
{ 
  \@@_prop_clear_new_map:n{#1} 
}
%    \end{macrocode}
% \end{function}
% \leavevmode
% \refstepcounter{subsection}
% \docsetnameref{\cs{CcoolHook}}
% \label{usage:cs:hook}
% \addcontentsline{toc}{subsection}{\protect\numberline{\thesubsection}\cs{CcoolHook}}
% \begin{function}{\CcoolHook}
%   \begin{description}
%   \item[Example] \cs{AfterEndEnvironment}|{theorem}{|\cs{CcoolHook}|}|
%   \end{description}
%    \begin{macrocode}
\NewDocumentCommand{\CcoolHook}{}{\c_empty_tl}
%    \end{macrocode}
% \end{function}
% \leavevmode
% \refstepcounter{subsection}
% \docsetnameref{\cs{CcoolLambda}}
% \label{usage:cs:lambda}
% \addcontentsline{toc}{subsection}{\protect\numberline{\thesubsection}\cs{CcoolLambda}}
% \begin{function}{\CcoolLambda}
%   \begin{arguments}
%   \item \meta{arg spec}
%   \item \meta{\doccceptcode}
%   \end{arguments}
%   \begin{description}
%   \item[Example] \cs{Ccool}|{ EvalAt = \CcoolLambda{(#1)} }|
%   \item[Semantics] Creates a lambda expression with \meta{\doccceptint} arguments for \meta{\doccceptcode}
%   \end{description}
%    \begin{macrocode}
\ProvideDocumentCommand \CcoolLambda { O{m} m }
{
  \@@_lambda:nn { #1 } { #2 }
}
%    \end{macrocode}
% \end{function}
% \leavevmode
% \refstepcounter{subsection}
% \docsetnameref{\cs{CcoolOption}}
% \label{usage:cs:option}
% \addcontentsline{toc}{subsection}{\protect\numberline{\thesubsection}\cs{CcoolOption}}
% \begin{function}{\CcoolOption}
%   \begin{arguments}
%   \item \meta{\doccceptkvl}
%   \end{arguments} 
%    \begin{macrocode}
\NewDocumentCommand{ \CcoolOption }
{ m }
{ 
  \keys_set:nn{ @@ }{#1} 
}
%    \end{macrocode}
% \end{function}
%    \begin{macrocode}
\keys_define:nn { @@ }
{
%    \end{macrocode}
% \leavevmode
% \refstepcounter{subsubsection}
% \docsetnameref{\pkgoptex}
% \label{usage:opt:ex}
% \addcontentsline{toc}{subsubsection}{\protect\numberline{\thesubsubsection}\pkgoptex}
% \DescribeOption{\pkgoptex}
% \begin{description}
% \item[Value] \texttt{eo\docpipe{}ee\docpipe{}ex\docpipe{}xo\docpipe{}xe\docpipe{}xx}
% \end{description}
%    \begin{macrocode}
Expans .multichoices:nn = { eo, ee, ex, xo, xe, xx }
{ \tl_gset_eq:NN \g@@_option_expans_tl \l_keys_choice_tl },
Expans .default:n = { xo },
Expans .initial:n = { xo },
%    \end{macrocode}
% \leavevmode
% \refstepcounter{subsubsection}
% \docsetnameref{\pkgoptfi}
% \label{usage:opt:fi}
% \addcontentsline{toc}{subsubsection}{\protect\numberline{\thesubsubsection}\pkgoptfi}
% \DescribeOption{\pkgoptfi}
% \begin{description}
% \item[Value] \meta{\doccceptpath}
% \end{description}
%    \begin{macrocode}
File .code:n = {
  \tl_gset:Nx \g@@_log_file_tl{#1}  
},
File .default:n = { \@@_sys_filename: },
File .initial:n = { \@@_sys_filename: },
%    \end{macrocode}
% \leavevmode
% \refstepcounter{subsubsection}
% \docsetnameref{\pkgoptin}
% \label{usage:opt:in}
% \addcontentsline{toc}{subsubsection}{\protect\numberline{\thesubsubsection}\pkgoptin}
% \DescribeOption{\pkgoptin}
% \begin{description}
% \item[Value] \meta{\doccceptcode}, with |####1| as the argument to be replaced
% \end{description} 
%    \begin{macrocode}
Inner .code:n={
  \@@_option_inner:n{#1}
  \exp_last_unbraced:Nf 
  \@@_make_ccool:nnnn
  {
    { \g@@_option_param_tl }
    { \g@@_option_inner_tl }
    { \g@@_option_separ_tl }
    { \g@@_option_outer_tl }
  }
},
Inner .value_required:n = false,
Inner .default:n = {####1},
Inner .initial:n = {####1},
%    \end{macrocode}
% \leavevmode
% \refstepcounter{subsubsection}
% \docsetnameref{\pkgoptpa}
% \label{usage:opt:pa}
% \addcontentsline{toc}{subsubsection}{\protect\numberline{\thesubsubsection}\pkgoptpa}
% \DescribeOption{\pkgoptpa}
% \begin{description}
% \item[Value] \meta{\docccepttl}
% \end{description} 
%    \begin{macrocode}
Param .code:n={
  \@@_option_param:n{#1}
  \exp_last_unbraced:Nf 
  \@@_make_ccool:nnnn
  {
    { \g@@_option_param_tl }
    { \g@@_option_inner_tl }
    { \g@@_option_separ_tl }
    { \g@@_option_outer_tl }
  }
},
Param .value_required:n = false,
Param .default:n = { Default },
Param .initial:n = { Default },
%    \end{macrocode}
% \leavevmode
% \refstepcounter{subsubsection}
% \docsetnameref{\pkgoptou}
% \label{usage:opt:ou}
% \addcontentsline{toc}{subsubsection}{\protect\numberline{\thesubsubsection}\pkgoptou}
% \DescribeOption{\pkgoptou}
% \begin{description}
% \item[Value] \meta{\doccceptcode}, with |####1| as the argument to be replaced
% \end{description}  
%    \begin{macrocode}
Outer .code:n={
  \@@_option_outer:n{#1}
  \exp_last_unbraced:Nf 
  \@@_make_ccool:nnnn
  {
    { \g@@_option_param_tl }
    { \g@@_option_inner_tl }
    { \g@@_option_separ_tl }
    { \g@@_option_outer_tl }
  }
},
Outer .value_required:n = false,
Outer .default:n = { \ensuremath{####1} },
Outer .initial:n = { \ensuremath{####1} },
%    \end{macrocode}
% \leavevmode
% \refstepcounter{subsubsection}
% \docsetnameref{\pkgoptse}
% \label{usage:opt:se}
% \addcontentsline{toc}{subsubsection}{\protect\numberline{\thesubsubsection}\pkgoptse}
% \DescribeOption{\pkgoptse}
% \begin{description}
% \item[Value] That of `separators' in \cite[Section 8 of \pkg{l3seq}]{interface3}
% \end{description}  
%    \begin{macrocode}
Separ .code:n={
  \@@_option_separ:n{#1}
  \exp_last_unbraced:Nf 
  \@@_make_ccool:nnnn
  {
    { \g@@_option_param_tl }
    { \g@@_option_inner_tl }
    { \g@@_option_separ_tl }
    { \g@@_option_outer_tl }
  }
},
Separ .value_required:n = false,
Separ .default:n = { {\ }and{\ } } { ,{\ } } { ,{\ }and{\ } },
Separ .initial:n = { {\ }and{\ } } { ,{\ } } { ,{\ }and{\ } },
%    \end{macrocode}
% \leavevmode
% \addtocounter{subsubsection}{1}
% \docsetnameref{\pkgoptwr}
% \label{usage:opt:wr}
% \addcontentsline{toc}{subsubsection}{\protect\numberline{\thesubsubsection}\pkgoptwr}
% \DescribeOption{\pkgoptwr}
% \begin{description}
% \item[Value] \meta{\doccceptbool}
% \end{description}
%    \begin{macrocode}
Write .code:n = {
  \bool_if:nTF{#1}
  {\@@_log_open:}
  {\@@_log_close:}
},
Write .value_required:n = false,
Write .default:n = \BooleanFalse,
Write .initial:n = \BooleanFalse
%    \end{macrocode}
%    \begin{macrocode}
}
%    \end{macrocode}
% \leavevmode
% \refstepcounter{subsection}
% \docsetnameref{\cs{CcoolRead}}
% \label{usage:cs:read}
% \addcontentsline{toc}{subsection}{\protect\numberline{\thesubsection}\cs{CcoolRead}}
% \begin{function}{\CcoolRead}
%   \begin{arguments}
%   \item  \meta{\doccceptpath}
%   \end{arguments} 
%   \begin{description}
%   \item[Semantics]\docfillblank
%     \begin{enumerate}
%     \item Reads the definitions in \meta{\doccceptpath}.
%     \item Writes to \file{ccool.log}: `read from \meta{\doccceptpath}'
%     \end{enumerate}
%   \end{description}  
% \end{function}
%    \begin{macrocode}
\NewDocumentCommand{\CcoolRead}
{o}
{
  \IfValueTF{#1}
  {\@@_log_read:e{#1}}
  {\@@_log_read:}
}
%    \end{macrocode}
% \leavevmode
% \refstepcounter{subsection}
% \docsetnameref{\cs{CcoolVers}}
% \label{usage:cs:vers}
% \addcontentsline{toc}{subsection}{\protect\numberline{\thesubsection}\cs{CcoolVers}}
% \begin{function}{\CcoolVers}
%   \begin{description}
%   \item[Semantics] Expands to the package's version
%   \end{description}  
%    \begin{macrocode}
\NewDocumentCommand{\CcoolVers}
{}
{\use:c{ver@ccool.sty}}
%    \end{macrocode}
% \end{function} 
% \section{Closing}
%    \begin{macrocode}
\ExplSyntaxOff
%    \end{macrocode}
% \end{implementation}
% 
% \iffalse
%</package> 
% \fi
% \Finale
\endinput
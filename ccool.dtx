% \iffalse meta-comment
% !TEX program  = pdfLaTeX
%<*internal> 
\iffalse
%</internal> 
%<*readme> 
----------------------------------------------------------------
##### ccool --- A key-value interface for generating commands
- Source repository: https://github.com/rogard/ccool
- Released under the LaTeX Project Public License v1.3c or later
- See http://www.latex-project.org/lppl.txt
----------------------------------------------------------------

%</readme> 
%<*internal> 
\fi
\def\nameofplainTeX{plain}
\ifx\fmtname\nameofplainTeX\else
\expandafter\begingroup
\fi
%</internal> 
%<*install> 
\input docstrip.tex
\keepsilent
\askforoverwritefalse
\preamble
----------------------------------------------------------------------------
ccool --- A key-value interface for generating commands
Released under the LaTeX Project Public License v1.3c or later
See http://www.latex-project.org/lppl.txt
----------------------------------------------------------------------------

\endpreamble
\postamble

Copyright (C) 2020 by Erwann Rogard

This work may be distributed and/or modified under the
conditions of the LaTeX Project Public License (LPPL), either
version 1.3c of this license or (at your option) any later
version.  The latest version of this license is in the file:

http://www.latex-project.org/lppl.txt

This work is "maintained" (as per LPPL maintenance status) by
Erwann Rogard.

This work consists of the file ccool.dtx and the derived files:
ccool.sty, and ccool.pdf.

\endpostamble
\generate{
  \file{\jobname.sty}{\from{\jobname.dtx}{package}}
}
%</install> 
%<install> \endbatchfile
%<*internal> 
\generate{
  \file{\jobname.ins}{\from{\jobname.dtx}{install}}
}
\nopreamble\nopostamble
\generate{
  \file{README.md}{\from{\jobname.dtx}{readme}}
}
\ifx\fmtname\nameofplainTeX
\expandafter\endbatchfile
\else
\expandafter\endgroup
\fi
%</internal> 
%<package> \NeedsTeXFormat{LaTeX2e}[2020/02/02]
%<package> \RequirePackage{etoolbox}[2019/09/21]
%<package> \RequirePackage{l3keys2e}[2020/03/06]
%<package> \RequirePackage{xparse}[2020/03/06]
%<package> \ProvidesExplPackage
%<package> {ccool}                                                                                      % Package name
%<package> {2020/04/16}                                                                                 % Release date
%<package> {2.4}                                                                                        % Release version
%<package> {cool --- A key-value interface for generating commands}                                     % Description
% 
%<*driver> 
\documentclass[full]{l3doc}
\listfiles
\usepackage{amsmath, amssymb}
\usepackage[english,french]{babel}
\usepackage{bookmark}
\usepackage{ccool}
\usepackage{fvextra}% csquotes should be loaded after fvextra
\usepackage[T1]{fontenc}% \char`[
\usepackage{microtype}
% ^^A\usepackage{pdfpages}
\usepackage{tabto}
\usepackage{tcolorbox}
\tcbuselibrary{listings, breakable}
\makeatletter
\newcommand*{\docsetnameref}{\def\@currentlabelname}%https://tex.stackexchange.com/questions/537751
\makeatother
\ExplSyntaxOn
\tl_gset:Nn \partname {Part}
\ExplSyntaxOff
\EnableCrossrefs
\CodelineIndex
\RecordChanges
% ^^A\AtEndDocument { \PrintChanges \PrintIndex }
\ExplSyntaxOn
\newenvironment{docabstract}[1]%https://latex.org/forum/viewtopic.php?t=12156
{\renewcommand{\abstractname}{#1}\begin{abstract}}
  {\end{abstract}} 
\cs_new:Nn \__erw_docu:n{\MakeUppercase #1}
\providecommand\docphsection[3]
{
  \leavevmode
  \refstepcounter{#1}
  \addcontentsline{toc}{#1}{\protect\numberline{\use:c{the#1}}#2}
  \docsetnameref{#2}
  \label{#3}
}
\cs_new:Nn \__ccool_docu:n{\MakeUppercase #1}
%^^A Sort--->
\providecommand\docU[1]{\exp_args:Nx \__erw_docu:n{#1}}
\providecommand\docarg[1]{\texttt{#1}} % fun[param] (macro) vs fun[arg] (eval)
\providecommand\docargnoval{\c_novalue_tl}
\providecommand\docassign[2]{#1~$\leftarrow$~#2}
\providecommand\docccept[1]{\textit{#1}}
\providecommand\doccceptargspec{arg~spec}
\providecommand\doccceptbool{boolean}
\providecommand\doccceptclist{clist}
\providecommand\doccceptcode{code}
\providecommand\doccceptcsname{cs name}
\providecommand\doccceptcsorcode{cs~or~code}
\providecommand\doccceptcs{cs}
\providecommand\doccceptempty{empty}
\providecommand\doccceptgroup{local~group}
\providecommand\doccceptint{integer}
\providecommand\doccceptitems{items}
\providecommand\doccceptkey{key}
\providecommand\doccceptkvl{keyval~list}
\providecommand\doccceptopt{option}
\providecommand\doccceptpath{path}
\providecommand\doccceptplaceh{placeholder}
\providecommand\doccceptpre{preamble}
\providecommand\doccceptprop{prop}
\providecommand\doccceptsep{separators}
\providecommand\doccceptseq{seq}
\providecommand\docccepttlvar{tl~var}
\providecommand\docccepttl{token~list}
\providecommand\docccepttok{token}
\providecommand\doccceptval{val}
\providecommand\docconv[1]{convention~\autoref{conv:#1}}
\providecommand\docdefaultfor{default~for~}
\providecommand\docenvdoc{\env{document}}
\providecommand\doceval[1]{\texttt{\char`\{}#1\texttt{\char`\}}}
\providecommand\docexpand[2]{#1~$\rightarrow$~#2}
\providecommand\docfillblank{\begin{minipage}[t]{\linewidth}\end{minipage}}
\providecommand\docissuedont{Don't: }
\providecommand\docissuedo{Do: }
\providecommand\docissuesymp{Symptom: }
\providecommand\doclist[1]{Listing~\ref{listing:#1}}
\providecommand\docnb{\noindent\textbf{NB}:~}
\providecommand\docoptd[1]{\texttt{\textless}#1\texttt{\textgreater}}
\providecommand\docopte[2]{\texttt{#1}\doceval{#2}}
\providecommand\docopto[1]{\texttt{[}#1\texttt{]}}
\providecommand\docpipe{\textbar}
\providecommand\docreflist[1]{Listing~\ref{listing:#1}}
\providecommand\docsee{See:~}
\providecommand\docstep[1]{step~\ref{step:#1}}
\providecommand\doctip{\noindent\textbf{Tip}:~}
\providecommand\docvers[2]{v#1.#2}
\providecommand\docwarn{\noindent\textbf{Warning}:~}
\providecommand\docxparsed[1][~argument]{`d'-type~#1}
\providecommand\docxparsee[1][~argument]{`e'-type~#1}
\providecommand\docxparsem[1][~argument]{`m'-type~#1}
\providecommand\docxparseo[1][~argument]{`o'-type~#1}
\providecommand\pkgkey{key$_{i}$}
\providecommand\pkgobj[1]{object identified by #1}
\providecommand\pkgoptex{\docarg{Expans}}
\providecommand\pkgoptfi{\docarg{File}}
\providecommand\pkgoptin{\docarg{Inner}}
\providecommand\pkgoptions{kvl0}
\providecommand\pkgoptou{\docarg{Outer}}
\providecommand\pkgoptpad{\docarg{Default}}%default
\providecommand\pkgoptpa{\docarg{Param}}
\providecommand\pkgoptse{\docarg{Separ}}
\providecommand\pkgoptwr{\docarg{Write}}
\providecommand\pkgparap{\texttt{+}}%append
\providecommand\pkgparex{\texttt{*}}%expand
\providecommand\pkgparhe{tl$_{1}$}%head
\providecommand\pkgparin{code$_{1}$}%inner
\providecommand\pkgparkvlnxt{kvl$_{2}$}
\providecommand\pkgparkvl{kvl$_{1}$}
\providecommand\pkgparou{code$_{2}$}%outer
\providecommand\pkgparpa{tl$_{2}$}%param
\providecommand\pkgparsiii{tl$_{5}$}
\providecommand\pkgparsii{tl$_{4}$}
\providecommand\pkgparsi{tl$_{3}$}
\providecommand\pkgpars{\Arg{\pkgparsi}\docpipe\Arg{\pkgparsi}\Arg{\pkgparsii}\docpipe\Arg{\pkgparsi}\Arg{\pkgparsii}\Arg{\pkgparsiii}}%separ
\providecommand\pkgparta{tl$_{6}$}%tail
\providecommand\pkgsep[1]{\Arg{#1}}
\providecommand\pkgval{val$_{i}$}
\providecommand\pkgwrite{\cs{Ccool}\docarg{\meta{\pkgparpa}}\docopte{i}{\meta{\pkgparin}}\Arg{\pkgparkvl}}
%^^A Sort<---
\ExplSyntaxOff
\begin{document}
\DocInput{\jobname.dtx}
\end{document}
%</driver> 
% \fi
% 
% \GetFileInfo{\jobname.sty}
% \begin{documentation}
%   \title{The \pkg{ccool} package\thanks{^^A
%   This file describes version \fileversion, last revised \filedate.^^A
% }^^A
% }
%   \author{Erwann Rogard\thanks{firstname dot lastname AusTria gmail dot com}}
%   
%   \date{Released \filedate}
%   
%   \maketitle
%   
%   \selectlanguage{english}
%   \begin{docabstract}{Abstract}
%     The package \pkg{ccool} for \LaTeX~provides a \textit{key-value} interface,\nameref{usage:cs:ccool}, meant to facilitate the generation of commands.
%     Optional parameters that control the processing of
%     the input and its expansion are set to their most likely usage.
%     This can be used to encode notational conventions
%     (such as \docexpand{\cs{Real}}{}|\mathbb{R}|) at the point where they are introduced in the \env{document} (``Let $\mathbb{R}$ denote real numbers'').
%     Polymorphic commands can be generated by parameterizing the keys
%    (for instance, one parameter value for style, another for a property).
%     User input to \cs{Ccool} can optionally be serialized.
%     This can useful for typesetting documents sharing the same notation.
%   \end{docabstract}
%   
%   \selectlanguage{french}
%   \begin{docabstract}{Résumé}
%     L'extension~\pkg{ccool} pour \LaTeX~met à disposition une interface de type \textit{clé-valeur}, \nameref{usage:cs:ccool},
%     destinée à faciliter la géneration de commandes. Les 
%     paramètre optionnels contrôlant le traitement de ces \textit{clé-valeur} sont fixés
%     par défaut pour répondre aux besoins courants. Ceci peut-être utilisé pour 
%     la command-isation des conventions de notation (\docexpand{\cs{Reel}}{} |\mathbb{R}|),
%     au point dans le \env{document} où elles sont introduites (``Soit~$\mathbb{R}$~les nombres réels.'').
%     Des commandes polymorphes peuvent être générées, en associant aux clés un paramètre
%     (par exemple, une valeur pour le style typographique, une autre pour la description du concept associé).
%     En option, les instructions passées à cette interface peuvent être sauvegardées,
%     ce qui peut être utile pour la rédaction de documents faisant appel à des conventions typographiques communes.
%   \end{docabstract}
%   \selectlanguage{english}
%   
%   \tableofcontents 
%   
%   \part{Usage}\label{part:usage}
%   ^^A   \VerbatimFootnotes
%   
%   \setcounter{section}{0}
%   \label{usage:conv}
%   \addcontentsline{toc}{section}{\protect\numberline{\thesection}Convention}
%   \section*{Convention}
%   \begin{enumerate}[label={\emph{\alph*)}}]
%   \item \label{conv:expl3} Loosely, those of \cite{interface3}, for example as to the meaning of \meta{\docccepttl}.
%   \item \label{conv:xparse} Those of \cite{xparse}, for example \docopto{arg} is a \docccept{\docxparseo}. 
%   \item \docassign{\meta{X}}{\docarg{Y}}:  set \meta{X} to |Y|
%   \item \docexpand{\cs{X}}{\docarg{Y}}: \cs{X} expands to |Y|
%   \item If unspecified, the environment in which a macro is to be used is \docenvdoc.
%   \end{enumerate}
%   
%   \refstepcounter{section}
%   \label{usage:load} 
%   \addcontentsline{toc}{section}{\protect\numberline{\thesection}Loading the package}
%   \begin{function}{\usepackage}
%     \begin{syntax}
%       \cs{usepackage}\doceval{\pkg{ccool}}
%     \end{syntax}
%     \begin{description}
%     \item[Requirement]\docfillblank
%       \begin{enumerate}
%       \item \file{ccool.sty} is in the path of the \LaTeX~engine. See \autoref{part:other}, \autoref{other:support}.
%       \item Put in the~\docccept{\doccceptpre}
%       \end{enumerate}
%     \end{description}  
%   \end{function}
%   
%  \docphsection{section}{\cs{Ccool}}{usage:cs:ccool}
%   \begin{function}{\Ccool}
%     \begin{syntax}
%       \cs{Ccool}\docopto{\meta{\pkgparhe}}\docoptd{\meta{\pkgparpa}}\docopte{c}{\meta{\pkgparin}}\Arg{\pkgparkvl}\pkgparap\pkgparex\docopte{s}{\meta{\doccceptsep}}\docopte{c}{\meta{\pkgparou}}\docopto{\meta{\pkgparta}}
%     \end{syntax}where \meta{\doccceptsep} is either of: \Arg{\pkgparsi}, \Arg{\pkgparsi}\Arg{\pkgparsii}, and \Arg{\pkgparsi}\Arg{\pkgparsii}\Arg{\pkgparsiii}.
%     \begin{description}
%     \item[Semantics] See \autoref{ccool:core}-\ref{ccool:wr}.
%     \end{description}
%   \end{function}
%   
%   \subsection{Core feature}
%   \label{ccool:core} 
%   \nameref{usage:cs:ccool}\Arg{\pkgparkvl} creates for each \meta{\pkgkey}|=|\meta{\pkgval},
%   the command \cs{\meta{\pkgkey}}, according to the following algorithm:
%   \begin{enumerate}[label=\emph{\arabic*)}]
%   \item \docassign{\meta{\pkgval} }
%     { \cs{function}\Arg{\pkgval} }\label{step:val}
%   \item Creates \cs{\meta{\pkgkey}}, such that \docexpand{ \cs{\meta{\pkgkey} } }{\meta{\pkgval}},
%     \label{step:key}
%   \end{enumerate}where \cs{function} is controled by \doccceptopt~\nameref{usage:opt:in}.
%   For instance, the side effect of |\Ccool{ Real = \mathbb{R} }| is \docexpand{\cs{Real}}{\docarg{\textbackslash{}mathbb\{R\}}}.
%   To be sparingly used, \docccept{\doccceptopt}~\nameref{usage:opt:ex} controls the way \meta{\pkgkey}~and \meta{\pkgval}~are expanded.
%   
%   See \nameref{usage:cs:lambda} to allow command \cs{\meta{\pkgkey}} to take arguments.
%   
%   \subsection{Process the \docccept{\pkgval}'s}
%   \label{ccool:in} 
%   \nameref{usage:cs:ccool}\ \docopte{c}{\meta{\pkgparin}}\Arg{\pkgparkvl} is identical to the \nameref{ccool:core},
%   except it overrides \nameref{usage:opt:in}.
%   
%   In our example, if multiple number systems are defined with \cs{Ccool} (natural, reals, \dots),
%   it is more efficient to omit |\mathbb{.}| inside \meta{\pkgval} and, instead,
%   use |c{\mathbb{#1}}|, where |#1| means ``parameter to be replaced''.
%   
%   \subsection{Append to a hook}
%   \label{ccool:ap} 
%   \nameref{usage:cs:ccool}\Arg{\pkgparkvl}\pkgparap~is identical to the \nameref{ccool:core}, except it
%   repeats after \nameref{usage:cs:hook}.
%   This is useful to make the side effect persist after a \docccept{\doccceptgroup} (such as \env{theorem}).
%   
%   \subsection{Expand the \docccept{\pkgval}'s}
%   \label{ccool:ex} 
%   \nameref{usage:cs:ccool}\Arg{\pkgparkvl}\pkgparex~supplements the \nameref{ccool:core} with 
%   the expansion of the \meta{\pkgval}'s using typesetting rules controlled by \docccept{\doccceptopt}~\nameref{usage:opt:se} 
%   and~\nameref{usage:opt:ou}. The first are \docccept{\doccceptsep} applied to the \meta{\pkgval}'s to form a \docccept{\docccepttl},
%   and the second a function applied to the latter. 
%
%   They can be overriden inline by appending further \docopte{s}{\meta{\doccceptsep}}~and~\docopte{c}{\meta{\pkgparou}}, respectively,
%   to the list of arguments.
%   
%   \subsection{Head}
%   \label{ccool:he}
%   \nameref{usage:cs:ccool}\docopto{\meta{\pkgparhe}}\Arg{\pkgparkvl}~expands \meta{\pkgparhe} and executes the \nameref{ccool:core}.
%   
%   There may be situations where it is convenient to pass \meta{\pkgparhe}~as \doccceptempty.
%   
%   \subsection{Tail}
%   \label{ccool:ta}
%   \nameref{usage:cs:ccool}\Arg{\pkgparkvl}\docopto{\meta{\pkgparta}}\Arg{\pkgparkvlnxt}~is identical to
%   \nameref{usage:cs:ccool}\Arg{\pkgparkvl} followed by \nameref{usage:cs:ccool}\docopto{\meta{\pkgparta}}\Arg{\pkgparkvlnxt}.
%   
%   The combination of \nameref{ccool:core}, \nameref{ccool:he}, and \nameref{ccool:ta} allows to integrate typesetting and
%   the creation of commands.
%   
%   \subsection{Parameterize the \docccept{\pkgkey}'s}
%   \label{ccool:pa}
%   \nameref{usage:cs:ccool}\docoptd{\meta{\pkgparpa}}\Arg{\pkgparkvl} is identical to the  \nameref{ccool:core},
%   except \meta{\pkgkey} is replaced by \meta{\pkgkey\docoptd{\pkgparpa}}.
%   The default parameter, that implicit in \meta{\pkgkey},  is controlled by \nameref{usage:opt:pa}.
%   In our example, \meta{\pkgparpa} could be \docarg{Style}.
%   
%   \subsection{Write}
%   \label{ccool:wr} 
%   If \docccept{\doccceptopt}~\nameref{usage:opt:wr}~is set to \cs{BooleanTrue}, the \nameref{ccool:core}
%   is supplemented with the code  written to a file, whose \doccceptpath~is controlled by \docccept{\doccceptopt}~\nameref{usage:opt:fi}.
%
%  \docphsection{section}{\cs{CcoolClear}}{usage:cs:clear}
%   \begin{function}{\CcoolClear}
%     \begin{syntax}
%       \cs{CcoolClear}\docoptd{\meta{\pkgparpa}}\Arg{\doccceptclist}
%     \end{syntax}
%     \begin{description}
%     \item[Semantics] Clears all \cs{\meta{\pkgkey\docoptd{\pkgparpa}}}'s
%     \end{description}
%   \end{function}
%   
% \docphsection{section}{\cs{CcoolHook}}{usage:cs:hook}
% \begin{function}{\CcoolHook}
%     \begin{syntax}
%       \cs{CcoolHook}
%     \end{syntax}
%     \begin{description}
%     \item[Semantics] No side effect or expansion
%     \end{description}
%   \end{function}
%   
% \docphsection{section}{\cs{CcoolLambda}}{usage:cs:lambda}
%   \begin{function}{\CcoolLambda}
%     \begin{syntax}
%       \cs{CcoolLambda}\docopto{\meta{\doccceptargspec}}\Arg{\doccceptcode},
%     \end{syntax} where \docccept{\doccceptargspec} is by default an \docccept{\docxparseo}.
%     \begin{description}
%     \item[Example] \cs{Ccool}|{ EvalAt = \CcoolLambda{(#1)} }|
%     \item[Semantics] Returns a command of type  \cs{DeclareDocumentCommand}\cite{xparse},
%     \end{description}
%   \end{function}
%   
% \docphsection{section}{\cs{CcoolOption}}{usage:cs:option}
%   \begin{function}{\CcoolOption}
%     \begin{syntax}
%       \cs{CcoolOption}\Arg{\doccceptkvl}
%     \end{syntax}
%     \begin{description}
%     \item[Semantics] Controls the default behavior of \nameref{usage:cs:ccool}. 
%     \end{description}
%   \end{function}
%   
%   \docphsection{subsection}{\pkgoptex}{usage:opt:ex}
%   \DescribeOption{\pkgoptex}
%   \begin{description}
%   \item[Also see] \autoref{part:impl}, \nameref{impl:opt:ex}
%   \item[Semantics] See \nameref{ccool:core}
%   \item[Syntax] \texttt{eo\docpipe{}ee\docpipe{}ex\docpipe{}xo\docpipe{}xe\docpipe{}xx}
%   \end{description}
%   
%   \docphsection{subsection}{\pkgoptfi}{usage:opt:fi}
%   \DescribeOption{\pkgoptfi}
%   \begin{description}
%   \item[Also see] \autoref{part:impl}, \nameref{impl:opt:fi}
%   \item[Semantics] See \nameref{ccool:wr}
%   \item[Syntax] \meta{\doccceptpath}
%   \end{description}
%   
%   \docphsection{subsection}{\pkgoptin}{usage:opt:in}
%   \DescribeOption{\pkgoptin}
%   \begin{description}
%   \item[Also see] \autoref{part:impl}, \nameref{impl:opt:in}
%   \item[Semantics] See \nameref{ccool:in}
%   \item[Syntax] \meta{\doccceptcode}, with |####1| as the argument to be replaced
%   \end{description}
%   
%   \docphsection{subsection}{\pkgoptpa}{usage:opt:pa}
%   \DescribeOption{\pkgoptpa}
%   \begin{description}
%   \item[Also see] \autoref{part:impl}, \nameref{impl:opt:pa}
%   \item[Semantics] See \nameref{ccool:pa}
%   \item[Syntax] \meta{\docccepttl}
%   \end{description}
%   
%   \docphsection{subsection}{\pkgoptou}{usage:opt:ou}
%   \DescribeOption{\pkgoptou}
%   \begin{description}
%   \item[Also see] \autoref{part:impl}, \nameref{impl:opt:ou}
%   \item[Semantics] See \nameref{ccool:ex}
%   \item[Syntax] \meta{\doccceptcode}, with |####1| as the argument to be replaced
%   \end{description}
%   
%   \docphsection{subsection}{\pkgoptse}{usage:opt:se}
%   \DescribeOption{\pkgoptse}
%   \begin{description}
%   \item[Also see] \autoref{part:impl}, \nameref{impl:opt:se}
%   \item[Semantics] See \nameref{ccool:ex}
%   \item[Syntax] That of \docccept{\doccceptsep} in \cite[Section 8 of \pkg{l3seq}]{interface3}
%   \end{description}
%   
%   \docphsection{subsection}{\pkgoptwr}{usage:opt:wr}
%   \DescribeOption{\pkgoptwr}
%   \begin{description}
%   \item[Also see] \autoref{part:impl}, \nameref{impl:opt:wr}
%   \item[Semantics] See \nameref{ccool:wr}
%   \item[Syntax] \meta{\doccceptbool}
%   \end{description}
%   
%   \docphsection{section}{\cs{CcoolRead}}{usage:cs:read}
%   \begin{function}{\CcoolRead}
%     \begin{syntax}
%       \cs{CcoolRead}\docopto{\meta{\doccceptpath}}
%     \end{syntax}
%     \begin{description}
%     \item[Also see] \autoref{part:impl}, \nameref{impl:cs:read}
%     \item[Semantics]\docfillblank
%       \begin{enumerate}
%       \item Reads the definitions in \meta{\doccceptpath}.
%       \item Writes to \file{ccool.log}: `read from \meta{\doccceptpath}'
%       \end{enumerate}
%     \end{description}  
%   \end{function}
%   
%   \docphsection{section}{\cs{CcoolVers}}{usage:cs:vers}
%   \begin{function}{\CcoolVers}
%     \begin{syntax}
%       \cs{CcoolVers}
%     \end{syntax}
%     \begin{description}
%     \item[Semantics] \docexpand{}{the package's version}
%     \end{description}  
%   \end{function}
%   
%   \section{Do's and dont's}
%   \label{usage:dosdont}
%   
%   \begin{enumerate}[label=\emph{\arabic*)}]
%   \item \docfillblank
%     \begin{itemize}
%     \item[\docissuedont] |$|\meta{\pkgkey}|<x$|.
%     \item[\docissuedo] |$|\cs{\meta{\pkgkey}}|{<}x$|
%     \end{itemize}
%   \item\docfillblank
%     \begin{itemize}
%     \item[\docissuedont] |[a, b)|
%     \item[\docissuedo] |{[}a, b{)}|
%     \end{itemize}
%   \item \docfillblank
%     \begin{itemize}
%     \item[\docissuedont] |\cal F|.
%     \item[\docissuedo] |\cal{F}| or |\mathcal{F}|
%     \end{itemize}
%   \item \docfillblank
%     \begin{itemize}
%     \item[\docissuedont] |\[x_0,x\]|
%     \item[\docissuedo] |\left[x_0,x\right]|
%     \end{itemize}
%   \item \docfillblank
%     \begin{itemize}
%     \item[\docissuedont] Use \docccept{\docxparsed[]} or \docccept{\docxparsee[]}~arguments for \nameref{usage:cs:lambda}
%     \item[\docissuedo] Use only \docccept{\docxparsem[]} and \docccept{\docxparseo[]}~arguments\NB{todo}{encode in \nameref{impl:cs:lambda}}
%     \end{itemize}
%   \item Also see \autoref{part:other}, \autoref{other:issue}
%   \end{enumerate}
%   
%   \clearpage
%   \part{Listing}\label{part:listing}
%   
%   \newtcblisting[auto counter]
%   {listing}[2][]{
%   noparskip,
%   breakable,
%   colback=white,
%   colframe=black,
%   opacitybacktitle=.8,%
%   fonttitle=\bfseries,
%   title={Listing~\thetcbcounter. #1},
%   arc=0pt,
%   outer arc=0pt,
%   boxrule=1pt,
%   listing and text,
%   #2}
%   
%   \docnb
%   \begin{enumerate}
%   \item These listings depend on the \cs{usepackage} statements of the source file's \env{documentation}
%   \item Statements involving \nameref{usage:opt:wr} or \nameref{usage:cs:clear}
%    affect only the output of listings that come after that in which they appear. The demarcation is indicated by |%^^||A--->| and |%^^||A<---|, where applicable
%   \end{enumerate}
%   \phantomsection\addcontentsline{toc}{section}
%   {\ref{listing:vers}.\cs{CcoolVers}}
%   \iffalse
%<*guardlisting>   
%   \fi
\begin{listing}[\cs{CcoolVers}]
  {label=listing:vers}
  \CcoolVers
\end{listing}
% \iffalse
%</guardlisting> 
% \fi
% 
% \phantomsection\addcontentsline{toc}{section}
% {\ref{listing:numbrsys:base}. ``Let~$\mathbb{N}$ and $\mathbb{R}$ denote\dots'' (start of the tutorial)}
% \iffalse
%<*guardlisting> 
% \fi
\begin{listing}[``Let~$\mathbb{N}$ and $\mathbb{R}$ denote\dots'' (start of the tutorial)]
  {label=listing:numbrsys:base, listing and text}
  Let~$\mathbb{N}$ and $\mathbb{R}$ denote the natural and real numbers.
\end{listing}
% \iffalse
%</guardlisting> 
% \fi
% 
% \phantomsection\addcontentsline{toc}{section}
% {\ref{listing:numbrsys:cmd}. Equivalent to \ref{listing:numbrsys:base}, with \cs{NewDocumentCommand}}
% \iffalse
%<*guardlisting> 
% \fi
\begin{listing}[ Equivalent to \ref{listing:numbrsys:base}, with \cs{NewDocumentCommand}]
  {label=listing:numbrsys:cmd, listing and text}
  \DeclareDocumentCommand\Nat{}{\mathbb{N}}
  \DeclareDocumentCommand\Real{}{\mathbb{R}}
  Let~$\Nat$ and $\Real$ denote the natural and real numbers.
\end{listing}
% \iffalse
%</guardlisting> 
% \fi
% 
% \phantomsection\addcontentsline{toc}{section}
% {\ref{listing:numbrsys:in}.  Equivalent to \ref{listing:numbrsys:cmd}, with \cs{Ccool}}
% \iffalse
%<*guardlisting> 
% \fi
\begin{listing}[ Equivalent to \ref{listing:numbrsys:cmd}, with \cs{Ccool}]
  {label=listing:numbrsys:in, listing and text}
  %^^A--->
  \Ccool c{\mathbb{#1}}{ Nat = {N}, Real = {R} }
  Let~$\Nat$ and $\Real$~denote the natural and real numbers.
  %^^A<---
  \CcoolClear
\end{listing}
% \iffalse
%</guardlisting> 
% \fi
% 
% \NB{todo}{c{\mathbb{#1}}{ Nat = {N}, Real = {R} }*}
% \phantomsection\addcontentsline{toc}{section}
% {\ref{listing:numbrsys:ex}.  Equivalent to \ref{listing:numbrsys:in}, with expansion}
% \iffalse
%<*guardlisting> 
\fi
\begin{listing}[ Equivalent to \ref{listing:numbrsys:in}, with expansion]
  {label=listing:numbrsys:ex, listing and text}
  %^^A--->
  \Ccool[Let~]
  c{\mathbb{#1}}{ Nat = {N}, Real = {R} }*s{{~\rm{and}~}}
  [~denote the natural and real numbers.]{}
  %^^A<---
  \CcoolClear
\end{listing}
% \iffalse
%</guardlisting> 
% \fi
% 
% \phantomsection\addcontentsline{toc}{section}
% {\ref{listing:numbrsys:pa}.  Equivalent to \ref{listing:numbrsys:in}, parameterized (end of the tutorial)}
% \iffalse
%<*guardlisting> 
% \fi
\begin{listing}[ Equivalent to \ref{listing:numbrsys:in}, parameterized (end of the tutorial)]
  {label=listing:numbrsys:pa, listing and text}
  %^^A--->
  \Ccool<Style>c{\mathbb{#1}}{ Nat = {N}, Real = {R} }
  [Let $\Nat<Style>$ and $\Real<Style>$ denote the natural and real numbers.]{}
  %^^A<---
  \CcoolClear<Style>
\end{listing}
% \iffalse
%</guardlisting> 
% \fi
% 
% \phantomsection\addcontentsline{toc}{section}
% {\ref{listing:se}. Separators.}
% \iffalse
%<*guardlisting> 
% \fi   ^^A% Separ: spaces betw. inner and outer brackets matter!->
\begin{listing}[Separators]
  {label=listing:se}
  %^^A--->
  \CcoolOption{
    Separ={{\ \char`@\ }{\ \%\ }{\ \char`@\ }}}
  \Ccool{ X = x, Y = y }*[\\]
  { X = x, Y = y, Z = z }*[\\]
  { X = x, Y = y }*s{{\ \&\ }}[\\]
  { X = x, Y = y }*s{{\ \&\ }{,\ }}[\\]
  { X = x, Y = y, Z = z }*s{{\ \&\ }}[\\]
  { X = x, Y = y, Z = z }*s{{\ \&\ }{,\ }}[\\]
  { X = x, Y = y, Z = z }*s{{\ \&\ }{,\ }{\ \&\ }}\\
  %^^A<---
  \CcoolClear
\end{listing}
% \iffalse
%</guardlisting> 
% \fi
% 
% \phantomsection\addcontentsline{toc}{section}
% {\ref{listing:hw:i}. Hello, world! (testing)}
% \iffalse
%<*guardlisting> 
% \fi
\begin{listing}[Hello, world! (testing)]
  {label=listing:hw:i}
  \CcoolOption{ Write = \BooleanTrue }
  %^^A--->
  \CcoolOption{Separ = {{}{.}{.}}, Outer = {####1}}
  \Ccool
  <Test>{ KeyA = {.}, KeyB = {!}, KeyC = {\%} }[]
  <Test>{ KeyD = {d}, KeyE = {\%} }[]
  <Test>c{\{#1\}}{ KeyF = {H}, KeyG = {e}, KeyH = {l} }*[]
  <Test>{ KeyI = {\%}, KeyJ = {\%}, KeyK = {\%} }[.\{l\}.\{o\}]
  <Test>{ KeyL = {l}, KeyM = {\char`[}, KeyN = {\char`]} }[]
  <Test>{ KeyO = {o}, KeyP = {\%}, KeyQ = {\%} }[{,\ }]
  <Test>{ KeyR = {w}, KeyS = {o}, KeyT = {r} }*
  s{{}{}{}}c{{\char`[}#1}[]
  <Test>{ KeyU = {\%}, KeyV = {\%}, KeyW = {\%} }[]
  <Test>{ KeyX = {\%}, KeyY = {\%}, KeyZ = {\KeyB<Test>} }\nobreak
  \KeyL<Test>\KeyD<Test>\KeyZ<Test>\KeyN<Test>\\
  %^^A<---
  \CcoolOption{ Write = \BooleanFalse }
  \CcoolClear
\end{listing}
% \iffalse
%</guardlisting> 
% \fi
% 
% \phantomsection\addcontentsline{toc}{section}
% {\ref{listing:hw:ii}. \docreflist{hw:i} read from file}
% \iffalse
%<*guardlisting> 
% \fi
\begin{listing}[\docreflist{hw:i} read from file]
  {label=listing:hw:ii}
  %^^A--->
  \CcoolRead
  \KeyF<Test>\KeyA<Test>\nobreak
  \KeyG<Test>\KeyA<Test>\nobreak
  \KeyH<Test>\KeyA<Test>\nobreak
  \KeyH<Test>\KeyA<Test>\nobreak
  {\{}\nobreak\KeyO<Test>{\}},{\ }\nobreak
  \KeyM<Test>\KeyR<Test>\nobreak
  \KeyO<Test>\nobreak
  \KeyT<Test>\nobreak
  \KeyL<Test>\nobreak
  \KeyD<Test>\nobreak
  \KeyZ<Test>\nobreak
  \KeyN<Test>\nobreak
  %^^A<---
  \CcoolClear
\end{listing}
% \iffalse
%</guardlisting> 
% \fi
% 
% \phantomsection\addcontentsline{toc}{section}{\ref{listing:proba:i}. Probability space}
% \iffalse
%<*guardlisting> 
% \fi
\begin{listing}[Probability space]
  {label=listing:proba:i}
  \CcoolOption{ Write = \BooleanTrue }
  %^^A--->
  \Ccool[Let~]
  { Space = \Omega, Field = \mathcal{F}, Meas = \mathcal{P} }
  *s{{,}}c{$\{#1\}$}
  [~denote the probability space, where~]{ PowerSet = { 2^{\Space} } }
  [$\Field\subset \PowerSet$.]
  {}
  \CcoolOption{ Write = \BooleanFalse }
  %^^A<---
  \CcoolClear
\end{listing}
% \iffalse
%</guardlisting> 
% \fi
% 
% \phantomsection\addcontentsline{toc}{section}{\ref{listing:proba:ii}. \docreflist{proba:i} read from file}
% \iffalse
%<*guardlisting> 
% \fi
\begin{listing}[\docreflist{proba:i} read from file]
  {label=listing:proba:ii}
  %^^A--->
  \CcoolRead \tab $\Omega$ $\Field$ $\Meas$
  %^^A<---
  \CcoolClear
\end{listing}
% \iffalse
%</guardlisting> 
% \fi
% 
% \phantomsection\addcontentsline{toc}{section}{\ref{listing:mvt:i}. Mittelwertsatz f\"ur $n$ Variable.}
% \iffalse
%<*guardlisting> 
% \fi
\begin{listing}[{Mittelwertsatz f\"ur $n$ Variable\cite[17.3]{tcolbox}}]
  {label=listing:mvt:i}
  \CcoolOption{ Write = \BooleanTrue }
  %^^A--->
  \newtheorem{theorem}{Theorem}
  \AfterEndEnvironment{theorem}{\CcoolHook}
  \Ccool c{\mathbb{#1}}
  { N = { N } , R = { R } }+[]
  { Grad = { \operatorname{grad} } }+
  [\begin{theorem}
    [Mittelwertsatz f\"ur $n$ Variable]Es~sei~]
    { OffMenge = {D}, Ci = {C^{1}}, Strecke = { \left[x_0,x\right] } }+
    [$n\in\N$,~$\OffMenge\subseteq\N^n$ eine offene Menge und $f\in\Ci(\OffMenge,\R)$.
    Dann gibt es auf jeder Strecke $\Strecke\subset\OffMenge$ einen Punkt $\xi\in\Strecke$,~]
    { Steig = { \frac{ f(x)-f(x_0) }{ x-x_0 } }, Punkt = { \xi } }+
    [so dass gilt
    \begin{equation*}
      \Steig = \Grad f(\Punkt)^{\top}
    \end{equation*}
  \end{theorem}]
  {}
  (Check: $\N$, $\Punkt$)
  %^^A<---
  \CcoolOption{ Write = \BooleanFalse }
  \CcoolClear
\end{listing}
% \iffalse
%</guardlisting> 
% \fi
% 
% \phantomsection\addcontentsline{toc}{section}{\ref{listing:mvt:ii}. \docreflist{mvt:i} read from file}
% \iffalse
%<*guardlisting> 
% \fi
\begin{listing}[\docreflist{mvt:i} read from file]
  {label=listing:mvt:ii}
  %^^A--->
  \CcoolRead \tab $\N$ $\R$ $\OffMenge$ $\Ci$ $\Strecke$ 
  %^^A<---
  \CcoolClear
\end{listing}
% \iffalse
%</guardlisting> 
% \fi
% 
% \phantomsection\addcontentsline{toc}{section}{\ref{listing:poly:real:i}. Polynôme}
% \iffalse
%<*guardlisting> 
% \fi
\begin{listing}[Polynôme]
  {label=listing:poly:real:i}
  \CcoolOption{ Write = \BooleanTrue }
  % ^^A--->
  \Ccool c{\mathbb{#1}}{ Nat = {N}, Reel = {R} }
  [Soient~]
  { PolyR = \CcoolLambda[o]{\Reel\IfValueT{#1}{_#1}[X] } }
  [$\PolyR[n]$ et $\PolyR$, les familles de polyn\^omes sur $\Reel$, de degr\'e $n$ et leur union pour $n \in \Nat$, respectivement. ]
  {}
  % ^^A<---
  \CcoolOption{ Write = \BooleanFalse }
  \CcoolClear
\end{listing}
% \iffalse
%</guardlisting> 
% \fi
% 
% \phantomsection\addcontentsline{toc}{section}{\ref{listing:poly:real:ii}. \docreflist{poly:real:i} read from file}
% \iffalse
%<*guardlisting> 
% \fi
\begin{listing}[\docreflist{poly:real:i} read from file]
  {label=listing:poly:real:ii}
  %^^A--->
  \CcoolRead \tab $\PolyR[n]$ et $\PolyR$
  %^^A<---
  \CcoolClear
\end{listing}
% \iffalse
%</guardlisting> 
% \fi
% 
% \phantomsection\addcontentsline{toc}{section}{\ref{listing:poly:gen:i}. Same as \docreflist{poly:real:i}, but arbitrary number system}
% \iffalse
%<*guardlisting> 
% \fi
\begin{listing}[Same as \docreflist{poly:gen:i}, but arbitrary number system]
  {label=listing:poly:gen:i}
  \CcoolOption{ Write = \BooleanTrue }
%^^A--->
\Ccool c{\mathbb{#1}}{ Corps = {K}, Nat = {N}, Reel = {R} }
[Soient~]
{
  Poly = \CcoolLambda[om]{#2\IfValueT{#1}{_#1}[X] },
  PolyR = \CcoolLambda[o]{\Poly[#1]{\Reel}}
}
[$\Poly[n]{\Corps}$ et $\Poly{\Corps}$, les familles de polyn\^omes sur $\Corps$, de degr\'e $n$ et leur union pour $n \in \Nat$, respectivement. En particulier,
ils sont d\'enot\'es $\PolyR[n]$ et $\PolyR$, pour $\Corps=\Reel$.]
{}
%^^A<---
\CcoolOption{ Write = \BooleanFalse }
\CcoolClear
\end{listing}
% \iffalse
%</guardlisting> 
% \fi
% 
% \phantomsection\addcontentsline{toc}{section}{\ref{listing:poly:real:ii}. \docreflist{poly:gen:i} read from file}
% \iffalse
%<*guardlisting> 
% \fi
\begin{listing}[\docreflist{poly:gen:i} read from file]
  {label=listing:poly:gen:ii}
  %^^A--->
  \CcoolRead \tab $\PolyR[n]$ et $\PolyR$
  %^^A<---
  \CcoolClear
\end{listing}
% \iffalse
%</guardlisting> 
% \fi
%
% \phantomsection\addcontentsline{toc}{section}{\ref{listing:fonc:i}. Fonction et fonctionelle}
% \iffalse
%<*guardlisting> 
% \fi
\begin{listing}[Fonction et fonctionelle]
  {label=listing:fonc:i}
  \CcoolOption{ Write = \BooleanTrue }
  %^^A--->
  \Ccool{ EvalAt = \CcoolLambda{(#1)}, ApplyOp = \CcoolLambda[mm]{#1[#2]} }
  [Supposons une fonction $f\EvalAt{t}$, et \'etudions le probl\`eme o\`u la fonctionnelle $\ApplyOp{S}{f}$ est donn\'ee par\dots]{}
  %^^A<---
  \CcoolOption{ Write = \BooleanFalse }
  \CcoolClear
\end{listing}
% \iffalse
%</guardlisting> 
% \fi
% 
% \phantomsection\addcontentsline{toc}{section}{\ref{listing:fonc:ii}. \docreflist{fonc:i} read from file}
% \iffalse
%<*guardlisting> 
% \fi
\begin{listing}[\docreflist{fonc:i} read from file]
  {label=listing:fonc:ii}
  %^^A--->
  \CcoolRead \tab $f\EvalAt{t}$, $\ApplyOp{S}{f}$
  %^^A<---
  \CcoolClear
\end{listing}
% \iffalse
%</guardlisting> 
% \fi
% 
% \phantomsection\addcontentsline{toc}{section}{\ref{listing:cusum:i}. CUSUM statistic.}
% \iffalse
%<*guardlisting> 
% \fi
\begin{listing}[CUSUM statistic\cite{ccool-thesis}]
  {label=listing:cusum:i}
  \CcoolOption{ Write = \BooleanTrue }
  %^^A--->
  \newtheorem{definition}{Definition}
  \AfterEndEnvironment{definition}{\CcoolHook}  
  \Ccool{ SuchThat = { ;~ }, Time = { t }, Process = { \xi }, StopT = { T }, EvalAt = \CcoolLambda{(#1)}  }
  [The CUSUM statistic process and the corresponding one-sided CUSUM stopping time are defined as follows:
  \begin{definition}\label{the CUSUM statistic}. Let~]
    { Scale = { \lambda }, Real = {\mathcal{R}} }+*s{{~\in~}}[~and~]
    { CUSUMthresh = { \nu } }+*c{$#1\in\Real^{+}$.}
    [~Define the following processes:]
    { LogWald = { u },  CUSUMst = { \StopT_{c} }, CUSUM = { y }, LogWaldInf = { m } }+
    [\begin{enumerate}
    \item{$\LogWald_{\Time}\EvalAt{ \Scale } = \Scale\Process_{\Time} - \frac{1}{2}\Scale^2\Time$;
        $\LogWaldInf_{\Time}\EvalAt{ \Scale } = \inf_{ 0\le s \le \Time }\CUSUM_{s} \EvalAt{ \Scale }$.}
    \item{$\CUSUM_{\Time}\EvalAt{ \Scale } = \LogWaldInf_{\Time}\EvalAt{ \Scale } - \LogWald_{\Time}\EvalAt{ \Scale }\ge0$, which is the CUSUM statistic process.}
    \item{$\CUSUMst \EvalAt{ \Scale, \LogWaldInf } = \inf\left[ \Time \ge 0 \SuchThat \CUSUM_{\Time}\EvalAt{\Scale} \ge \LogWaldInf \right]$, which is the CUSUM stopping time.} \end{enumerate}\end{definition}\par]{}
  
  (Check: $\Scale$, $\CUSUM$)
  %^^A<---
  \CcoolOption{ Write = \BooleanFalse }
  \CcoolClear
\end{listing}
% \iffalse
%</guardlisting> 
% \fi
% 
% \phantomsection\addcontentsline{toc}{section}{\ref{listing:cusum:ii}. \docreflist{cusum:i} read from file}
% \iffalse
%<*guardlisting> 
% \fi
% \begin{listing}[\docreflist{cusum:i} read from file]
%   {label=listing:cusum:ii}
% ^^A--->
%   \CcoolRead \tab $\Time $ $\Process$ $\Scale$ $\Real$ $\CUSUMthresh$ $\LogWald$  $\CUSUMst$ $\CUSUM$ $\LogWaldInf$ 
% ^^A<---
% \CcoolClear
% \end{listing}
% \iffalse
%</guardlisting> 
% \fi
% 
% \clearpage
% \part{Other}\label{part:other}
% 
% \section{Acknowledgment}\label{other:acknowl} 
% 
% This work has benefited from Q\&A's from the \LaTeX community\cite{user-erw}\cite{p158966}.
% Specific attributions are made throughout this document.
% 
% \section{Genealogy}\label{other:geneal}
% 
% \textquote
% [\pkg{cool}\cite{cool}]
% {Give commands the ability to contain the mathematical meaning while retaining the typesetting versatility}.
% The addition of `c', in \pkg{ccool}, is for \emph{custom}. With hinsdight it is restrictive to
% describe \pkg{ccool} as a tool for encoding mathematical convention.
% 
% \section{Install}\label{other:install}
% \begin{enumerate}[label=\emph{\arabic*)}]
% \item Compile \file{ccool.dtx} (under Unix, \texttt{\$tex ccool.dtx})
% \item Put the generated \file{ccool.sty} in the search path of the \LaTeX engine
% \end{enumerate}
% 
% \section{Issue}\label{other:issue}
% 
% \begin{enumerate}[label=\emph{\arabic*)}]
% \item
%   \begin{description}
%   \item[\docissuedont] |Inner=\{####1\}|
%   \item[\docissuesymp] \cs{CcoolRead} fails
%   \item[\docissuedo] |Inner={\char`{####1\char`}}|
%   \end{description}
% \end{enumerate}
% 
% \section{Support}\label{other:support}
% 
% This package is available from \url{https://www.ctan.org/pkg/ccool} and \url{https://github.com/rogard/ccool}.
% 
% \section{Testing}\label{other:testing}
% 
% \subsection{Technicality}
% Not possible to compile-check the expansion of a certain class of macros against predefined values\cite{a-534100}. 
% Instead, one can visually check \autoref{part:listing}, as generated in \autoref{other:install} on one's own machine,
% against that \href{https://github.com/rogard/ccool}{of the repository} for the same version.
% 
% 
% \subsection{Platform}
% \begin{enumerate}[label=\emph{\roman*)}]
% \item 
%   ^^A uname -a
%   \begin{Verbatim}[breaklines=true]
%     Linux laptop 4.15.0-20-generic #21-Ubuntu SMP Tue Apr 24 06:16:15 UTC 2018 x86_64 x86_64 x86_64 GNU/Linux
%   \end{Verbatim}
%   \label{plat:lin}
% \end{enumerate}
% 
% \subsection{Engine}
% \begin{enumerate}[label=\emph{\alph*)}]
% \item 
%   \begin{Verbatim}[breaklines=true]
%     pdfTeX 3.14159265-2.6-1.40.20 (TeX Live 2019)
%   \end{Verbatim}
%   \label{eng:tlxviiii:pdf}
% \item 
%   \begin{Verbatim}[breaklines=true]
%     pdfTeX 3.14159265-2.6-1.40.21 (TeX Live 2020)
%   \end{Verbatim}
%   \label{eng:tlxx:pdf}
% \item
%   \begin{Verbatim}[breaklines=true]
%     LuaHBTeX, Version 1.12.0 (TeX Live 2020)
%   \end{Verbatim}
%   \label{eng:tlxx:lua}
% \item
%   \begin{Verbatim}[breaklines=true]
%     XeTeX 3.14159265-2.6-0.999992 (TeX Live 2020)
%   \end{Verbatim}
%   \label{eng:tlxx:xe}
% \end{enumerate}
% 
% \subsection{Results}
% 
% \begin{enumerate}[label=\emph{\arabic*)}]
% \item \pkg{ccool} \docvers{1}{8} compiles satisfactorily on platform \ref{plat:lin} and engine \ref{eng:tlxviiii:pdf}
% \item \pkg{ccool} \docvers{1}{8} compiles satisfactorily on platform \ref{plat:lin} and engine \ref{eng:tlxx:pdf}
% \item \pkg{ccool} \docvers{1}{9} compiles satisfactorily on platform \ref{plat:lin} and engines \ref{eng:tlxx:pdf} and \ref{eng:tlxx:lua}
% \item \pkg{ccool} \docvers{2}{0} compiles satisfactorily on platform \ref{plat:lin} and engines \ref{eng:tlxx:pdf},  \ref{eng:tlxx:lua}, and \ref{eng:tlxx:xe}
% \item \pkg{ccool} \docvers{2}{1} compiles satisfactorily on platform \ref{plat:lin} and engines \ref{eng:tlxx:pdf},  \ref{eng:tlxx:lua}, and \ref{eng:tlxx:xe}
% \item \pkg{ccool} \docvers{2}{3} compiles satisfactorily on platform \ref{plat:lin} and engines \ref{eng:tlxx:pdf},  \ref{eng:tlxx:lua}, and \ref{eng:tlxx:xe}
% \end{enumerate}
% 
% \subsection{Other}
% Check \cite{ccool-thesis} for testing \pkg{ccool} with \cls{llncs}
%
% \section{To do}
%
% \begin{enumerate}[label=\emph{\arabic*)}]
% \item \docccept{\docU{\doccceptplaceh}} passed to \autoref{part:impl}~\nameref{impl:cs:option} should be |#1| not |####1|
% \item \nameref{impl:cs:option} should behave in away similar to that described in \autoref{part:usage}~\autoref{usage:opt:wr}
% \end{enumerate}
% 
% \docphsection{section}{References}{other:bib}
% \begin{thebibliography}{1}
% \bibitem{cool} Nick Setzer {\em The \pkg{cool} package}, 2005, \url{https://www.ctan.org/pkg/cool}
% \bibitem{interface3} The \LaTeX3 Project Team {\em The \LaTeX3 interfaces}, 2019,
%   \url{http://ftp.math.purdue.edu/mirrors/ctan.org/macros/latex/contrib/l3kernel/interface3.pdf}
% \bibitem{tcolbox} Thomas F. Sturm {\em The \pkg{tcolorbox} package}, 2019,
%   \url{http://www.texdoc.net/texmf-dist/doc/latex/tcolorbox/tcolorbox.pdf}
% \bibitem{xparse} The \LaTeX3 Project Team {\em The \pkg{xparse} package}, 2020,
%   \url{http://ftp.math.purdue.edu/mirrors/ctan.org/macros/latex/contrib/l3packages/xparse.pdf}
% \bibitem{ccool-thesis} Erwann Rogard and Olympia Hadjiliadis {\em Typesetting a math thesis with \pkg{ccool}}, 2020,
%   \url{https://github.com/rogard/ccool/blob/master/thesis.pdf}
% \bibitem{user-erw} \url{https://tex.stackexchange.com/users/112708/erwann?tab=questions}
% \bibitem{a-188053} \href{https://tex.stackexchange.com/users/17423/sean-allred}{@{}sean-allred}'s answer to
%   ``How to create lambda expressions?'', \url{https://tex.stackexchange.com/a/188053/112708}
% \bibitem{a-534100} \href{https://tex.stackexchange.com/users/73/joseph-wright}{@{}joseph-wright}'s answer to
%   ``Checking a function's expansion against a string'', \url{https://tex.stackexchange.com/a/534100}
% \bibitem{a-536597} \href{https://tex.stackexchange.com/users/73317/frougon}{@{}frougon}'s answer to ``Journaling calls to a function \textdel{taking inline code as argument}'', \url{https://tex.stackexchange.com/a/536620}
% \bibitem{p158966}\cs{Ccool}, extension à \LaTeX~à vocation mathématique, \url{http://forum.mathematex.net/latex-f6/ccool-extension-latex-a-vocation-mathematique-t17314.html}
% \end{thebibliography}
% 
% \changes{\docvers{1}{0}}
% {2020/03/08}
% {Initial version}
% \changes{\docvers{1}{1}}
% {2020/04/04}
% {Rearrange: much of the implementation}
% \changes{\docvers{1}{1}}
% {2020/04/04}
% {Replace: \Arg{kvl_{2}}~by~\docoptd{kvl_{2}}~given that option type \texttt{G} not recommended\cite{xparse}}
% \changes{\docvers{1}{1}}
% {2020/04/04}
% {Replace: \cs{OopsOptions}~by~\cs{OopsOption}}
% \changes{\docvers{1}{1}}
% {2020/04/04}
% {Replace: \docarg{GenericObject}~by~\docarg{Name}}
% \changes{\docvers{1}{1}}
% {2020/04/04}
% {Replace: \docarg{Separators}~by~\docarg{Separ}}
% \changes{\docvers{1}{1}}
% {2020/04/04}
% {Add:\cs{OopsTest}}
% \changes{\docvers{1}{1}}
% {2020/04/04}
% {Add:\cs{OopsRestore}}
% \changes{\docvers{1}{1}}
% {2020/04/04}
% {Add: \docarg{Save}}
% \changes{\docvers{1}{1}}
% {2020/04/04}
% {Delete: Listing 1-5 from \docvers{1}{0}}
% \changes{\docvers{1}{1}}
% {2020/04/04}
% {Add: Listing~1., 2., 3., 4., 6., and 9.}
% \changes{\docvers{1}{1}}
% {2020/04/04}
% {Fix: apparent anomaly in \docvers{1}{0}'s Listing~4, see \docreflist{hw:i}}
% \changes{\docvers{1}{2}}
% {2020/04/06}
% {Add: optional \pkgparex to \cs{OopsNew} as instruction to expand \pkgparkvl}
% \changes{\docvers{1}{2}}
% {2020/04/06}
% {Replace: \cs{OopsClear}\Arg{\pkgparpa} by \cs{OopsClear}\docopto{\meta{\doccceptkvl}}}
% \changes{\docvers{1}{2}}
% {2020/04/06}
% {Delete: \meta{kvl_{2}}~and~\meta{code_{2}}}
% \changes{\docvers{1}{2}}
% {2020/04/06}
% {Delete: \cs{OopsTest}}
% \changes{\docvers{1}{2}}
% {2020/04/06}
% {Delete: Listing~2-3 from \docvers{1}{1}.}
% \changes{\docvers{1}{2}}
% {2020/04/06}
% {Replace: \cs{Save} by \cs{Write}}
% \changes{\docvers{1}{2}}
% {2020/04/06}
% {Replace: \cs{Restore} by \cs{Read}}
% \changes{\docvers{1}{3}}
% {2020/04/06}
% {Replace: \cs{OopsNew} by \cs{Oops}}
% \changes{\docvers{1}{3}}
% {2020/04/06}
% {Replace: \Arg{\pkgparpa} and \docopto{\meta{\pkgparpa}} by \docoptd{\meta{\pkgparpa}} }
% \changes{\docvers{1}{4}}
% {2020/04/10}
% {Replace: \docopte{s}{\Arg{\pkgparsi}\Arg{\pkgparsii}\Arg{\pkgparsiii}}
% by \docopte{s}{\Arg{\pkgparsi}\docpipe\Arg{\pkgparsi}\Arg{\pkgparsii}\docpipe\Arg{\pkgparsi}\Arg{\pkgparsii}\Arg{\pkgparsiii}}}
% \changes{\docvers{1}{4}}
% {2020/04/10}
% {Add:optional \pkgparap to \cs{OopsNew} to make side effects presist beyond local group}
% \changes{\docvers{1}{4}}
% {2020/04/10}
% {Add: \cs{OopsHook}}
% \changes{\docvers{1}{4}}
% {2020/04/10}
% {Delete: Listing~1., and 2. }
% \changes{\docvers{1}{4}}
% {2020/04/10}
% {Add: Listing~1., 2., and 3. }
% \changes{\docvers{1}{4}}
% {2020/04/10}
% {Add: \docarg{Expans} (for debugging' sake, but...) }
% \changes{\docvers{1}{4}}
% {2020/04/10}
% {Add: \autoref{usage:dosdont} }
% \changes{\docvers{1}{4}}
% {2020/04/10}
% {Add: \cs{OopsDebug} }
% \changes{\docvers{1}{5}}
% {2020/04/10}
% {Delete: dependence on \pkg{datetime} }
% \changes{\docvers{1}{5}}
% {2020/04/10}
% {Add: \pkgoptfi }
% \changes{\docvers{1}{6}}
% {2020/04/10}
% {Rename: \pkg{oops} to \pkg{ccool} (better describes the purpose) }
% \changes{\docvers{1}{6}}
% {2020/04/10}
% {Rename: \cs{Oops} to \cs{Ccool}}
% \changes{\docvers{1}{6}}
% {2020/04/10}
% {Rename: \cs{OopsClear} to \cs{CcoolClear}}
% \changes{\docvers{1}{6}}
% {2020/04/10}
% {Rename: \cs{OopsDebug} to \cs{CcoolDebug}}
% \changes{\docvers{1}{6}}
% {2020/04/10}
% {Rename: \cs{OopsHook} to \cs{CcoolHook}}
% \changes{\docvers{1}{6}}
% {2020/04/10}
% {Rename: \cs{OopsOption} to \cs{CcoolOption}}
% \changes{\docvers{1}{6}}
% {2020/04/10}
% {Rename: \cs{OopsRead} to \cs{CcoolRead}}
% \changes{\docvers{1}{6}}
% {2020/04/10}
% {Add: Listing showing part of the preamble }
% \changes{\docvers{1}{7}}
% {2020/04/11}
% {Add: \docreflist{cusum:i} (CUSUM) }
% \changes{\docvers{1}{7}}
% {2020/04/11}
% {Add: Legends to listings }
% \changes{\docvers{1}{7}}
% {2020/04/11}
% {Delete: \cs{CcoolDebug} }
% \changes{\docvers{1}{7}}
% {2020/04/11}
% {Delete: Listing~5 from \docvers{1}{6} }
% \changes{\docvers{1}{8}}
% {2020/04/12}
% {Add: \hyperref[usage:cs:lambda]{\cs{CcoolLambda}} }
% \changes{\docvers{1}{8}}
% {2020/04/12}
% {Add: \docreflist{fonc:i}, \docreflist{fonc:ii} }
% \changes{\docvers{1}{8}}
% {2020/04/12}
% {Add: \cs{CcoolVers} }
% \changes{\docvers{1}{8}}
% {2020/04/12}
% {Add: \docreflist{vers} }
% \changes{\docvers{1}{9}}
% {2020/04/14}
% {Add: support for \LuaTeX }
% \changes{\docvers{1}{9}}
% {2020/04/14}
% {Move: from \autoref{part:usage} to \autoref{part:impl}, what is now that part's \autoref{impl:frontend}}
% \changes{\docvers{2}{0}}
% {2020/04/15}
% {Delete: \pkgoptfi's dependency on \pkg{texosquery} and \cs{pdfcreationdate}}
% \changes{\docvers{2}{0}}
% {2020/04/15}
% {Add: support for \XeTeX}
% \changes{\docvers{2}{0}}
% {2020/04/15}
% {Update: \cs{RequirePackage}, \cs{NeedsTeXFormat}'s second argument / TeX Live 2020}
% \changes{\docvers{2}{1}}
% {2020/04/17}
% {Replace: \hyperref[usage:cs:lambda]{\cs{CcoolLambda}}'s optional integer argument (number of \texttt{m}'s)
% by a standard argument list}
% \changes{\docvers{2}{1}}
% {2020/04/17}
% {Replace: \hyperref[usage:par:pa]{\meta{\pkgparpa}}'s position within \cs{Ccool}'s argument list, from first to second. Greater versatility }
% \changes{\docvers{2}{1}}
% {2020/04/17}
% {Replace: as~the~default~of~\pkgoptpa,~\docarg{Math}~by~\pkgoptpad}
% \changes{\docvers{2}{1}}
% {2020/04/17}
% {Replace: \docarg{Name} by \nameref{usage:opt:pa} }
% \changes{\docvers{2}{1}}
% {2020/04/17}
% {Add:Listings 3, 4, 5, 6, 7, 8, and 9}
% \changes{\docvers{2}{2}}
% {2020/04/20}
% {Remove: \% from listings}
% \changes{\docvers{2}{2}}
% {2020/04/20}
% {Replace: part of the abstract's with more straighforward descriptions based on input from forum participtants}
% \changes{\docvers{2}{3}}
% {2020/05/16}
% {Replace: for \cs{Ccool}, \docopte{i}{} by \docopte{c}{}}
% \changes{\docvers{2}{3}}
% {2020/05/16}
% {Rearranged: \nameref{usage:cs:ccool}'s subsections. Previously, by argument. Now, by feature. }
% \changes{\docvers{2}{3}}
% {2020/05/16}
% {Replace: In \docstep{key}, the created command's implementation, from \cs{ProvideDocumentCommand} to \cs{DeclareDocumentCommand}}
% \changes{\docvers{2}{3}}
% {2020/05/16}
% {Complete: \docreflist{poly:real:i}}
% \changes{\docvers{2}{3}}
% {2020/05/16}
% {Add: \docreflist{poly:real:ii}, \docreflist{poly:gen:i}, and \docreflist{poly:gen:ii}}
% \changes{\docvers{2}{3}}
% {2020/05/16}
% {Remove: Listing showing part of the preamble }
% \changes{\docvers{2}{4}}
% {2020/05/16}
% {Fix: minor error in the listings (\cs{Real} rather than \cs{Reel}, hitherto unnoticed). }
% \changes{\docvers{2}{4}}
% {2020/05/16}
% {Remove: examples from \autoref{part:usage},~\autoref{usage:cs:ccool}, as redundant with \autoref{part:listing}~\docreflist{numbrsys:base}-\ref{listing:numbrsys:pa} }
%
% \PrintChanges
% \PrintIndex
% \clearpage
% \StopEventually{
% ^^A   \PrintChanges
% ^^A   \PrintIndex
% }
% \end{documentation}
% ^^A% Commented out to eliminate WARNING: Reference `doc/function//
% \begin{implementation}
%   \selectlanguage{english}
%   \part{Implementation}\label{part:impl}
%   
%   \iffalse
%<*package>   
%   \fi
%   \section{Opening}
%    \begin{macrocode}
%<@@=ccool>      
\ExplSyntaxOn
%    \end{macrocode}
% \section{\texttt{aux}}
% \begin{macro}{\@@_aux_inner_set:n}
%   \begin{arguments}
%   \item \meta{code}
%   \end{arguments}
%    \begin{macrocode}
\cs_new_protected:Nn \@@_aux_inner_set:n
{
  \cs_gset:Npn \@@_aux_inner:n ##1 {#1}
  \cs_generate_variant:Nn \@@_aux_inner:n { e }
}
%    \end{macrocode}
% \end{macro}
% \begin{macro}{\@@_aux_key:w }
%   \begin{arguments}
%   \item \meta{ key }
%   \item \meta{ value }
%   \end{arguments}
%    \begin{macrocode}
\cs_new_protected:Npn \@@_aux_key:w #1 = #2 \q_stop
{
  \seq_gput_right:Nx \g@@_aux_key_seq { \tl_trim_spaces:n{#1} }
}
%    \end{macrocode}
% \end{macro}
% \begin{macro}{\@@_aux_key:n }
%   \begin{arguments}
%   \item \meta{ key = value }
%   \end{arguments}
%    \begin{macrocode}
\cs_new_protected:Nn \@@_aux_key:n
{
  \@@_aux_key:w #1 \q_stop
}
%    \end{macrocode}
% \end{macro}
% \begin{macro}{\@@_aux_key:N }
%   \begin{arguments}
%   \item \meta{ seq }
%   \end{arguments}
%    \begin{macrocode}
\cs_new_protected:Nn \@@_aux_key:N 
{
  \seq_gclear_new:N \g@@_aux_key_seq
  \seq_map_function:NN #1 \@@_aux_key:n
}
%    \end{macrocode}
% \end{macro}
% \begin{macro}{\@@_aux_outer_set:n}
%   \begin{arguments}
%   \item \meta{ inline code }
%   \end{arguments}
%    \begin{macrocode}
\cs_new_protected:Nn \@@_aux_outer_set:n
{
  \cs_gset:Npn \@@_aux_outer:n ##1 {#1}
}
%    \end{macrocode}
% \end{macro}
% \begin{macro}{\@@_aux_prop:nn}
%    \begin{macrocode}
\prop_new:N \g@@_aux_prop
\cs_new_protected:Nn \@@_aux_prop:nn 
{
  \prop_gput:Nnn \g@@_aux_prop{#1}{#2}
}
\cs_generate_variant:Nn \@@_aux_prop:nn { eo, ee, ex, xo, xe, xx }
%    \end{macrocode}
% \end{macro}
% \begin{macro}{\@@_aux_prop:w}
%   \begin{arguments}
%   \item \meta{ key }
%   \item \meta{ value }
%   \end{arguments}
%    \begin{macrocode}
\tl_new:N \g@@_option_expans_tl
\cs_new_protected:Npn \@@_aux_prop:w #1 = #2 \q_stop
{
  \exp_args:Nx
  \use:c{@@_aux_prop:\g@@_option_expans_tl}
  { \tl_trim_spaces:n{#1} }
  { \@@_aux_inner:n{ \tl_trim_spaces:n{#2} } }
}
%    \end{macrocode}
% \end{macro}
% \begin{macro}{\@@_aux_prop:n}
%   \begin{arguments}
%   \item \meta{ key = value }
%   \end{arguments}
%    \begin{macrocode}
\cs_new_protected:Nn \@@_aux_prop:n
{
  \@@_aux_prop:w #1 \q_stop 
}
%    \end{macrocode}
% \end{macro}
% \begin{macro}{\@@_aux_prop:N}
%   \begin{arguments}
%   \item \meta{keyval list}
%   \end{arguments}
%    \begin{macrocode}
\cs_new_protected:Nn \@@_aux_prop:N
{
  \prop_gclear_new:N \g@@_aux_prop
  \seq_if_empty:NTF #1
  { \c_empty_tl }
  {
    \seq_map_function:NN #1 \@@_aux_prop:n
  }
}
%    \end{macrocode}
% \end{macro}
% \begin{macro}{\@@_aux_separ:nn}
%   \begin{arguments}
%   \item \meta{ int }
%   \item \meta{ tokens }
%   \end{arguments}
%    \begin{macrocode}
\cs_new:Nn \@@_aux_separ:nn
{
  \int_case:nnTF {#1}
  {
    {1}
    { \prg_replicate:nn{ 3 }{#2} }
    {2}
    {
      { \use_i:nn #2 }
      { \use_ii:nn #2 }
      { \use_i:nn #2 }
    }
    {3}{#2}
  }
  { \c_empty_tl }
  {
    \msg_error:nnnn { @@ }
    { separ }
    { \exp_not:N \_@@_aux_separ:nn }
    {#2}
  }
}
\cs_generate_variant:Nn \@@_aux_separ:nn { e }
%    \end{macrocode}
% \end{macro}
% \begin{macro}{\@@_aux_separ:n}
%   \begin{arguments}
%   \item \meta{ tokens }
%   \end{arguments}
%    \begin{macrocode}
\cs_new:Nn \@@_aux_separ:n
{
  \@@_aux_separ:en{ \tl_count:n{#1} }{#1}
}
%    \end{macrocode}
% \end{macro}
% \begin{macro}{\@@_aux_val:Nn}
%   \begin{arguments}
%   \item \meta{ seq }
%   \item \meta{ tl var name }
%   \end{arguments}
%    \begin{macrocode}
\cs_new_protected:Nn \@@_aux_val:Nn
{
  \seq_gclear_new:N \g@@_aux_val_seq
  \@@_seq_from_prop:NNn \g@@_aux_val_seq #1 { \@@_prop_name:n{#2} } 
}
%    \end{macrocode}
% \end{macro}
% \section{\texttt{lambda}}
% \begin{macro}{\@@_lambda:nn}\cite{a-188053}
%    \begin{macrocode}
\cs_new_protected:Npn \@@_lambda:nn #1 #2 
{
  \exp_args:NNx 
  \DeclareDocumentCommand \@@_lambda_expression 
  {#1}
  {#2}
  \@@_lambda_expression
}
%    \end{macrocode}
% \end{macro}
% \section{\texttt{log}}
% \begin{macro}{\@@_log_close:}
%    \begin{macrocode}
\iow_new:N \g@@_log_iow
\AtEndDocument{\iow_close:N \g@@_log_iow}
\bool_set_false:N \g@@_log_open_bool
\cs_new_protected:Nn \@@_log_close:
{
  \iow_close:N \g@@_log_iow
  \bool_gset_false:N \g@@_log_open_bool
}
%    \end{macrocode}
% \end{macro}
% \begin{macro}{\@@_log_open:}
%    \begin{macrocode}
\tl_new:N \g@@_log_file_tl
\cs_new_protected:Nn \@@_log_open:
{
  \tl_gset:Nx \g@@_log_to_tl{\g@@_log_file_tl}
  \iow_open:Nn \g@@_log_iow {\g@@_log_to_tl}
  \bool_gset_true:N \g@@_log_open_bool
}
%    \end{macrocode}
% \end{macro}
% \begin{macro}{\@@_log_read:n}
%   \begin{arguments}
%   \item \meta{path}
%   \end{arguments}
%    \begin{macrocode}
\cs_new_protected:Nn \@@_log_read:n
{
  \file_input:n{#1}
  \tl_log:n{read~from~#1}
}
\cs_generate_variant:Nn \@@_log_read:n { e }
%    \end{macrocode}
% \end{macro}
% \begin{macro}{\@@_log_read:}
%    \begin{macrocode}
\cs_new_protected:Nn \@@_log_read:
{
  \@@_log_read:e{\g@@_log_to_tl}
}
%    \end{macrocode}
% \end{macro}
% \begin{macro}{\@@_log_write:n}
%    \begin{macrocode}
\tl_new:N \g@@_log_to_tl
\cs_new_protected:Nn \@@_log_write:n
{
  \bool_if:nTF{ \g@@_log_open_bool }
  {
    \iow_now:Nn \g@@_log_iow {#1}
    \tl_log:n{ write~to~#1 }
  }
  { \msg_error:nnn{ @@ }{ iow }{ \g@@_log_iow }  }
}
\cs_generate_variant:Nn \@@_log_write:n { e }
%    \end{macrocode}
% \end{macro}
% \section{\texttt{make_key}}
% \begin{macro}{\@@_make_key:Nn}
%   \begin{arguments}
%   \item \meta{ token }
%   \item \meta{ key }
%   \end{arguments}
%    \begin{macrocode}
\cs_new_protected:Nn \@@_make_key:Nn 
{
  \exp_args:NNx
  \DeclareDocumentCommand{#1} 
  { D<>{\g@@_option_param_tl} }
  {
    \@@_prop_item:nn{##1}{#2}
  }
}
\cs_generate_variant:Nn \@@_make_key:Nn {c}
%    \end{macrocode}
% \end{macro}
% \begin{macro}{\@@_make_key:n}
%   \begin{arguments}
%   \item \meta{ key }
%   \end{arguments}
%    \begin{macrocode}
\cs_new_protected:Nn \@@_make_key:n
{
  \@@_make_key:cn{#1}{#1}
}
\cs_generate_variant:Nn \@@_make_key:n { e }
%    \end{macrocode}
% \end{macro}
% \begin{macro}{\@@_make_key:N}
%   \begin{arguments}
%   \item \meta{ seq }
%   \end{arguments}
%    \begin{macrocode}
\cs_new_protected:Nn \@@_make_key:N
{
  \seq_map_function:NN #1 \@@_make_key:e
}
%    \end{macrocode}
% \end{macro}
% \section{\texttt{make_ccool}}
% \begin{macro}{\@@_make_ccool_exp:nnn}
%    \begin{macrocode}
\cs_new_protected:Nn \@@_make_ccool_exp:nnn
{
  \@@_aux_val:Nn \g@@_aux_key_seq {#1}
  \@@_aux_outer_set:n{#3}
  \@@_aux_outer:n
  {
    \exp_args:NNf
    \@@_seq_use:Nn
    \g@@_aux_val_seq
    {#2}
  }
}
%    \end{macrocode}
% \end{macro}
% \begin{macro}{\@@_make_ccool_key:nnn}
%    \begin{macrocode}
\cs_new_protected:Nn \@@_make_ccool_key:nnn
{
  \@@_prop_if_exist:nTF{#1}
  { \c_empty_tl }
  { \@@_prop_new:n{#1} }
  \exp_args:No \@@_aux_inner_set:n{#2}
  \seq_set_from_clist:Nn \g@@_aux_keyval_seq {#3}
  \@@_aux_prop:N \g@@_aux_keyval_seq
  \@@_prop_append:Nn \g@@_aux_prop {#1}
  \@@_aux_key:N \g@@_aux_keyval_seq
  \@@_make_key:N \g@@_aux_key_seq
}
%    \end{macrocode}
% \end{macro}
% \begin{macro}{\@@_make_ccool_sideeffect:nnn}\cite{a-536597}
%    \begin{macrocode}
\cs_new_protected:Nn \@@_make_ccool_sideeffect:nnn
{
  \@@_make_ccool_key:nnn{#1}{#2}{#3}
  \bool_if:nTF{ \g@@_log_open_bool }
  {
    \@@_log_write:n
    {
      \begingroup
      \def \@@_log_entry { \Ccool<#1>c{#2}{#3} } \expandafter
      \endgroup \@@_log_entry      
    }
  }{\c_empty_tl}
}
%    \end{macrocode}
% \end{macro}
% \begin{macro}{\@@_make_ccool:nnnn}
%   \begin{arguments}
%   \item \meta{ token list }
%   \item \meta{ seq_{1} }
%   \item \meta{ seq_{2} } 
%   \item \meta{ prop }
%   \end{arguments}
%    \begin{macrocode}
\cs_new_protected:Npn \@@_make_ccool:nnnn #1 #2 #3 #4
{ 
  \exp_args:NNx \DeclareDocumentCommand \Ccool
  {%^^A      2    3         4 5  6    7 8              9
    +o D<>{#1} E{ c }{{#2}} m t+ s E{ s c }{{#3}{#4}} +o
  }
  {
    \IfValueT{##1}{##1}    
    \@@_make_ccool_sideeffect:nnn{##2}{##3}{##4}    
    \IfBooleanT{##6}
    {
      \@@_make_ccool_exp:nnn{##2}{##7}{##8}
    }
    \bool_if:nTF{##5}
    {
      \gappto{\CcoolHook}
      {
        \@@_make_ccool_sideeffect:nnn{##2}{##3}{##4}
      }
    }
    {\c_empty_tl}
    \IfValueT{##9}
    {
      \exp_not:n{ \Ccool[##9] }
    }
  }
}
%    \end{macrocode}
% \end{macro}
% \section{\texttt{msg}}
%    \begin{macrocode}
\msg_new:nnn {@@}{ generic }{#1}
\msg_new:nnn {@@}{ iow }{#1~is~closed~can't~write}
\msg_new:nnn {@@}{ keyonly }{#1~does~not~take~values;~keyval~is~#2}
\msg_new:nnn {@@}{ keywrong }{#1~does~not~recognize~key~#2}
\msg_new:nnn {@@}{ separ }{#1~expects~1~to~3~items,~#2}
\msg_new:nnn {@@}{ unset }{#1~unset}
%    \end{macrocode}
% \section{\texttt{option}}
% \begin{macro}{\@@_option_inner:n}
%   \begin{arguments}
%   \item \meta{code}
%   \end{arguments}
%    \begin{macrocode}
\cs_new_protected:Nn \@@_option_inner:n
{
  \tl_gset:Nn \g@@_option_inner_tl {#1}
}
\@@_option_inner:n
{
  \msg_warning:nnn{ @@ }{ unset }{ \exp_not:N \g@@_option_inner_tl }
}
%    \end{macrocode}
% \end{macro}
% \begin{macro}{\@@_option_param:n}
%   \begin{arguments}
%   \item \meta{token list}
%   \end{arguments}
%    \begin{macrocode}
\cs_new:Nn \@@_option_param:n
{
  \tl_gset:Nn \g@@_option_param_tl{#1}
}
\@@_option_param:n
{  
  \msg_error:nnx{ @@ }
  { generic }
  { \exp_not:N\g@@_option_param_tl~undefined }
}
%    \end{macrocode}
% \end{macro}
% \begin{macro}{\@@_option_outer:n}
%   \begin{arguments}
%   \item \meta{ inline code }
%   \end{arguments}
%    \begin{macrocode}
\cs_new_protected:Nn \@@_option_outer:n
{
  \tl_gset:Nn \g@@_option_outer_tl {#1}
}
\@@_option_outer:n
{
  \msg_warning:nnn{ @@ }{ unset }{ \exp_not:N \g@@_option_outer_tl }
}
%    \end{macrocode}
% \end{macro}
% \begin{macro}{\@@_option_separ:n}
%   \begin{arguments}
%   \item \Arg{ tl_{1} }\Arg{ tl_{2} }\Arg{ tl_{3} }
%   \end{arguments}
%    \begin{macrocode}
\cs_new_protected:Nn \@@_option_separ:n
{
  \cs_gset:Npn \g@@_option_separ_tl {#1}
}
\@@_option_separ:n
{
  \msg_warning:nnn{ @@ }{ unset }{ \exp_not:N \g@@_option_separ_tl }
}
%    \end{macrocode}
% \end{macro}
% \section{\texttt{prop}}
% \begin{macro}{\@@_prop_append:NN}
%   \begin{arguments}
%   \item \meta{ prop_{1} }
%   \item \meta{ prop_{2} }
%   \end{arguments}
%    \begin{macrocode}
\cs_new_protected:Npn \@@_prop_append:NN #1 #2
{
  \cs_set:Nn \@@_prop_append:nn
  {
    \prop_gput:Nnx #1 {##1}{ \prop_item:Nn #2{##1} }
  }
  \prop_map_function:NN #2 \@@_prop_append:nn
}
\cs_generate_variant:Nn \@@_prop_append:NN { cN }
%    \end{macrocode}
% \end{macro}
% \begin{macro}{\@@_prop_append:Nn}
%   \begin{arguments}
%   \item \meta{ prop }
%   \item \meta{ tl var name }
%   \end{arguments}
%    \begin{macrocode}
\cs_new_protected:Nn \@@_prop_append:Nn 
{
  \@@_prop_append:cN{ \@@_prop_name:n {#2} } #1
}
%    \end{macrocode}
% \end{macro}
% \begin{macro}{\@@_prop_clear_new:n}
%   \begin{arguments}
%   \item \meta{ tl var name }
%   \end{arguments}
%    \begin{macrocode}
\cs_new_protected:Nn \@@_prop_clear_new:n
{
  \exp_args:No \prop_clear_new:c{ \@@_prop_name:n {#1} }
}
%    \end{macrocode}
% \end{macro}
% \begin{macro}{\@@_prop_clear_new_map:n}
%   \begin{arguments}
%   \item \meta{ keyval list }
%   \end{arguments}
%    \begin{macrocode}
\cs_new_protected:Nn \@@_prop_clear_new_map:n
{
  \seq_set_from_clist:Nn \g@@_aux_key_seq {#1}
  \seq_map_function:NN \g@@_aux_key_seq \@@_prop_clear_new:n
}
%    \end{macrocode}
% \end{macro}
% \begin{macro}{\@@_prop_if_exist:nTF}
%   \begin{arguments}
%   \item \meta{tl_{1}}
%   \item \meta{tl_{2}}
%   \item \meta{tl_{3}}
%   \end{arguments}
%    \begin{macrocode}
\cs_new:Nn \@@_prop_if_exist:nTF 
{
  \prop_if_exist:cTF{ \@@_prop_name:n {#1} }{#2}{#3}
}
%    \end{macrocode}
% \end{macro}
% \begin{macro}{\@@_prop_item:nn}
%   \begin{arguments}
%   \item \meta{ tl var name }
%   \item \meta{ key }
%   \end{arguments}
%    \begin{macrocode}
\cs_new:Nn \@@_prop_item:nn
{
  \prop_item:cn { \@@_prop_name:n {#1} } {#2}
}
%    \end{macrocode}
% \end{macro}
% \begin{macro}{\@@_prop_name:n}
%   \begin{arguments}
%   \item \meta{ tl var name }
%   \end{arguments}
%    \begin{macrocode}
\cs_new:Npn \@@_prop_name:n #1{ @@_#1 }
%    \end{macrocode}
% \end{macro}
% \begin{macro}{\@@_prop_new:n}
%   \begin{arguments}
%   \item \meta{ tl var name }
%   \end{arguments}
%    \begin{macrocode}
\cs_new_protected:Nn \@@_prop_new:n 
{
  \prop_new:c{ \@@_prop_name:n {#1} }
}
%    \end{macrocode}
% \end{macro}
% \section{\texttt{seq}}
% \begin{macro}{\@@_seq_from_prop:NNn}
%   \begin{arguments}
%   \item \meta{ seq_{1} }
%   \item \meta{ seq_{2} } (keys)
%   \item \meta{ prop }
%   \end{arguments}
%    \begin{macrocode}
\cs_new_protected:Nn \@@_seq_from_prop:NNn
{
  \cs_set_protected:Nn \@@_seq_from_prop:n
  {
    \seq_gput_right:No #1 { \prop_item:cn{#3}{##1} }
  }
  \seq_map_function:NN #2 \@@_seq_from_prop:n
}
%    \end{macrocode}
% \end{macro}
% \begin{macro}{\@@_erw_seq_use:Nn}
%    \begin{macrocode}
%      \begin{arguments}
%      \item \meta{ seq }
%      \item \meta{ tokens }
%      \end{arguments}
\cs_new:Nn \@@_seq_use:Nn
{
  \exp_last_unbraced:NNf
  \seq_use:Nnnn #1
  \@@_aux_separ:n{#2}
}
%    \end{macrocode}
% \end{macro}
% \section{sys}\label{impl:sys}
% \begin{macro}{\@@_sys_date:}
%    \begin{macrocode}
\cs_new:Nn \@@_sys_date:
{
  \int_eval:n
  {
    \c_sys_year_int * 10000
    +\c_sys_month_int * 100
    +\c_sys_day_int *  1
  }
}
%    \end{macrocode}
% \end{macro}
% \begin{macro}{\@@_sys_date_hex:}
%    \begin{macrocode}
\cs_new:Nn \@@_sys_date_hex:
{\int_to_hex:n{\@@_sys_date:}}
%    \end{macrocode}
% \end{macro}
% \begin{macro}{\@@_sys_time:}
%    \begin{macrocode}
\cs_new:Nn \@@_sys_time:
{
  \int_eval:n
  {
    \c_sys_hour_int * 100 
    +\c_sys_minute_int * 1 
  }
}
%    \end{macrocode}
% \end{macro}
% \begin{macro}{\@@_sys_time_hex:}
%    \begin{macrocode}
\cs_new:Nn\@@_sys_time_hex:
{\int_to_hex:n{\@@_sys_time:}}
%    \end{macrocode}
% \end{macro}
% \begin{macro}{\@@_sys_filename:}
%    \begin{macrocode}
\cs_new:Nn\@@_sys_filename:
{
  \c_sys_jobname_str--
  \@@_sys_date_hex:--
  \@@_sys_time_hex:
}
%    \end{macrocode}
% \end{macro}
% \section{Front-end}\label{impl:frontend}
% \docphsection{subsection}{\cs{CcoolClear}}{impl:cs:read}
% \begin{macro}{\CcoolClear}
%    \begin{macrocode}
\NewDocumentCommand{ \CcoolClear }
{ D<>{\g@@_option_param_tl} }
{ 
  \@@_prop_clear_new_map:n{#1} 
}
%    \end{macrocode}
% \end{macro}
% \docphsection{subsection}{\cs{CcoolHook}}{impl:cs:hook}
% \begin{macro}{\CcoolHook}
%    \begin{macrocode}
\NewDocumentCommand{\CcoolHook}{}{\c_empty_tl}
%    \end{macrocode}
% \end{macro}
% \docphsection{subsection}{\cs{CcoolLambda}}{impl:cs:lambda}
% \begin{macro}{\CcoolLambda}
%    \begin{macrocode}
\ProvideDocumentCommand \CcoolLambda { O{m} m }
{
  \@@_lambda:nn { #1 } { #2 }
}
%    \end{macrocode}
% \end{macro}
% \docphsection{subsection}{\cs{CcoolOption}}{impl:cs:option}
% \begin{macro}{\CcoolOption}
%    \begin{macrocode}
\NewDocumentCommand{ \CcoolOption }
{ m }
{ 
  \keys_set:nn{ @@ }{#1}
%^^A  \bool_if:nTF{ \g@@_log_open_bool }
%^^A  {
%^^A    \@@_log_write:n
%^^A    {
%^^A      \begingroup
%^^A      \def \@@_log_entry { \CcoolOption{ #1 } \expandafter
%^^A        \endgroup \@@_log_entry      
%^^A      }
%^^A    }{\c_empty_tl}
%^^A  }
}
%    \end{macrocode}
% \end{macro}
%    \begin{macrocode}
\keys_define:nn { @@ }
{
%    \end{macrocode}
% \docphsection{subsection}{\pkgoptex}{impl:opt:ex}
% \DescribeOption{\pkgoptex}
%    \begin{macrocode}
Expans .multichoices:nn = { eo, ee, ex, xo, xe, xx }
{ \tl_gset_eq:NN \g@@_option_expans_tl \l_keys_choice_tl },
Expans .default:n = { xo },
Expans .initial:n = { xo },
%    \end{macrocode}
% \docphsection{subsection}{\pkgoptfi}{impl:opt:fi}
% \DescribeOption{\pkgoptfi}
%    \begin{macrocode}
File .code:n = {
  \tl_gset:Nx \g@@_log_file_tl{#1}  
},
File .default:n = { \@@_sys_filename: },
File .initial:n = { \@@_sys_filename: },
%    \end{macrocode}
% \docphsection{subsection}{\pkgoptin}{impl:opt:in}
% \DescribeOption{\pkgoptin}
%    \begin{macrocode}
Inner .code:n={
  \@@_option_inner:n{#1}
  \exp_last_unbraced:Nf 
  \@@_make_ccool:nnnn
  {
    { \g@@_option_param_tl }
    { \g@@_option_inner_tl }
    { \g@@_option_separ_tl }
    { \g@@_option_outer_tl }
  }
},
Inner .value_required:n = false,
Inner .default:n = {####1},
Inner .initial:n = {####1},
%    \end{macrocode}
% \docphsection{subsection}{\pkgoptpa}{impl:opt:pa}
% \DescribeOption{\pkgoptpa}
%    \begin{macrocode}
Param .code:n={
  \@@_option_param:n{#1}
  \exp_last_unbraced:Nf 
  \@@_make_ccool:nnnn
  {
    { \g@@_option_param_tl }
    { \g@@_option_inner_tl }
    { \g@@_option_separ_tl }
    { \g@@_option_outer_tl }
  }
},
Param .value_required:n = false,
Param .default:n = { Default },
Param .initial:n = { Default },
%    \end{macrocode}
% \docphsection{subsection}{\pkgoptou}{impl:opt:ou}
% \DescribeOption{\pkgoptou}
%    \begin{macrocode}
Outer .code:n={
  \@@_option_outer:n{#1}
  \exp_last_unbraced:Nf 
  \@@_make_ccool:nnnn
  {
    { \g@@_option_param_tl }
    { \g@@_option_inner_tl }
    { \g@@_option_separ_tl }
    { \g@@_option_outer_tl }
  }
},
Outer .value_required:n = false,
Outer .default:n = { \ensuremath{####1} },
Outer .initial:n = { \ensuremath{####1} },
%    \end{macrocode}
% \docphsection{subsection}{\pkgoptse}{impl:opt:se}
% \DescribeOption{\pkgoptse}
%    \begin{macrocode}
Separ .code:n={
  \@@_option_separ:n{#1}
  \exp_last_unbraced:Nf 
  \@@_make_ccool:nnnn
  {
    { \g@@_option_param_tl }
    { \g@@_option_inner_tl }
    { \g@@_option_separ_tl }
    { \g@@_option_outer_tl }
  }
},
Separ .value_required:n = false,
Separ .default:n = { {\ }and{\ } } { ,{\ } } { ,{\ }and{\ } },
Separ .initial:n = { {\ }and{\ } } { ,{\ } } { ,{\ }and{\ } },
%    \end{macrocode}
% \docphsection{subsection}{\pkgoptwr}{impl:opt:wr}
% \DescribeOption{\pkgoptwr}
%    \begin{macrocode}
Write .code:n = {
  \bool_if:nTF{#1}
  {\@@_log_open:}
  {\@@_log_close:}
},
Write .value_required:n = false,
Write .default:n = \BooleanFalse,
Write .initial:n = \BooleanFalse
%    \end{macrocode}
%    \begin{macrocode}
}
%    \end{macrocode}
% \docphsection{subsection}{\cs{CcoolRead}}{impl:cs:read}
% \begin{macro}{\CcoolRead}
%    \begin{macrocode}
\NewDocumentCommand{\CcoolRead}
{o}
{
  \IfValueTF{#1}
  {\@@_log_read:e{#1}}
  {\@@_log_read:}
}
%    \end{macrocode}
% \end{macro}
% \docphsection{subsection}{\cs{CcoolVers}}{impl:cs:vers}
% \begin{macro}{\CcoolVers}
%    \begin{macrocode}
\NewDocumentCommand{\CcoolVers}
{}
{\use:c{ver@ccool.sty}}
%    \end{macrocode}
% \end{macro} 
% \section{Closing}
%    \begin{macrocode}
\ExplSyntaxOff
%    \end{macrocode}
% \end{implementation}
% 
% \iffalse
%</package> 
% \fi
% \Finale
\endinput